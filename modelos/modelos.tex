%
% modificado a nov 3 de 2014
% incluye paup*
% se elimina (mr)modeltest
%
\chapter{Modelos de evoluci\'on}


\section*{Introducci\'on}
\index{Likelihood!modelos}

\label{ch:molecular}
En sistem\'atica molecular se enfrenta un conjunto de datos distinto  a los caracteres morfol\'ogicos:  un mismo tipo de estados (ACGT/GAP) se encuentra repetido a lo largo de toda la matriz de datos, para este tipo de datos moleculares, existen aproximaciones estad\'isticas, dado que el origen de muchos de los an\'alisis filogen\'eticos moleculares es en biolog\'ia molecular o en gen\'etica de poblaciones. Bajo la estimaci\'on estad\'istica se asume que un modelo genera las secuencias de ADN y a partir de tal modelo se estima qu\'e tanto se ajustan los datos a las hip\'otesis filogen\'eticas. Este procedimiento se conoce como \textbf{m\'axima verosimilitud} (\textit{Maximum Likelihood} en la literatura en ingl\'es).

En los modelos moleculares se asignan tanto la probabilidad de transformar una base cualquiera en otra base, incluida esa misma base, como las frecuencias de bases en el equilibrio. El c\'alculo de esas probabilidades est\'a influenciado por diferentes par\'ametros como por los criterios de selecci\'on; en principio, los par\'ametros deber\'ian ser estimados a partir  de los datos, aunque algunos autores usaron en un inicio modelos predefinidos de acuerdo a la opini\'on del investigador.

El modelo con el menor n\'umero de par\'ametros es conocido como JC o JC69, por los autores y el a\~no en que fue propuesto \citep{Jukes1969}, en este modelo se asume que la probabilidad de una base para transformarse en otra es siempre la misma y que la frecuencia de las bases es igual. A partir de este modelo se pueden agregar m\'as par\'ametros que hacen m\'as complejo el modelo en diferentes direcciones, ya sea al diferenciar entre los tipos de base (ya sea qu\'imicamente AG-CT, o cada una por separado), diferenciar las probabilidades basados en la proporci\'on de cada base, tener o no en cuenta o la variaci\'on dada por cada sitio al asumir que algunos sitios son m\'as propensos a cambiar que otros (funci\'on $\Gamma$) y, finalmente, si existe o no un reloj molecular.

La informaci\'on b\'asica se puede consultar en \cite{Felsenstein2004} 
o \cite{Swofford1996}, este \'ultimo es una referencia b\'asica para comprender el desarrollo de los distintos modelos. \cite{PageHolmes1998} ofrecen un cap\'itulo ilustrativo sobre el tema y si desea un tratamiento avanzado debe consultarse \cite{yang2006}. La idea general de modelos tambi\'en se aplica para la evoluci\'on prote\'ica, pero dado que casi siempre se tiene tanto la secuencia de amino\'acidos como la nucleot\'idica, esta \'ultima es la preferida al explotar m\'as directamente la informaci\'on (al menos a nivel de variaci\'on).

Dado que los valores de las probabilidades de los datos son muy peque\~nas, y que es m\'as f\'acil sumar los logaritmos que multiplicar los n\'umeros iniciales, para facilitar se utiliza el negativo del logaritmo natural de la verosimilitud, que es el valor tradicionalmente reportado por los programas as\'i como en la literatura.

Debido al aumento en el n\'umero de los par\'ametros, los modelos puedan explicar mejor los datos, lo que no implica que los modelos sean \emph{\textbf{m\'as}} realistas, por lo que la selecci\'on de un modelo no puede basarse en el aumento neto del valor de verosimilitud, sino si la mejora es (o no es) estad\'isticamente significativa, dado el n\'umero de par\'ametros usados en el modelo.\footnote{Es buena idea familiarizarse con algunas pruebas de hip\'otesis usando verosimilitud, por lo que puede consultar la descripci\'on en la \href{http://en.wikipedia.org/wiki/Neyman\%E2\%80\%93Pearson\_lemma}{wiki}.}

Existen dos aproximaciones b\'asicas para este problema. Dado que la mayor parte de los modelos son una especializaci\'on de otros modelos m\'as sencillos, es decir son modelos anidados, en los cuales el modelo m\'as sencillo es s\'olo un caso especial del modelo m\'as complejo, se puede hacer una prueba jer\'arquica de verosimilitud (hLRT\footnote
{\begin{equation}
\delta = -2 ln \frac{Max[L_{0}(modelo-nulo|datos]}{Max[L_{1}(modelo-alternativo|datos]} = 2 (ln L_{0} - ln L_{1}) 
\end{equation}
}) que compara la proporci\'on en la que se incrementa la verosimilitud al agregar el par\'ametro; se asume una distribuci\'on $\chi^2$  para la proporci\'on, tomando el n\'umero de par\'ametros extra como los grados de libertad. El problema de esta prueba es que no es claro si la distribuci\'on $\chi^2$ es v\'alida, y solo permite comparar modelos anidados; la prueba es un tanto conflictiva al comparar GTR con GTR+I o GTR+$\Gamma$; se asume que la forma como se agregan los par\'ametros no sesga el resultado, lo cual no es v\'alido; usted puede revisar esta afirmaci\'on usando programas como \Pname{JModelTest} o \Pname{PAUP*}.

Otra forma de selecci\'on del modelo \'optimo  est\'a basada en la medida de la informaci\'on, usando el criterio de Akaike\footnote{AIC= -2lnL + 2N, donde:\\lnL es el ajuste del modelo reportado y \\N el n\'umero de par\'ametros libres}, 
o la cantidad de informaci\'on bayesiana\footnote{BIC=-2lnL + Nlnn, donde: \\lnn, es el logaritmo natural de el costo de la secuencia}.  
En estos casos se calcula la cantidad de informaci\'on que sugiere la topolog\'ia dado el modelo y los par\'ametros de este. La ventaja es que se puede hacer la comparaci\'on entre modelos sin tener que tomar una secuencia particular y sin importar si los modelos est\'an anidados.

Para los c\'alculos, se puede usar \Pname{JModeltest}, o \Pname{PAUP*}, o \Pname{R + ape + PhyML} o \Pname{JModeltest + PhyML}. 
 
\Pname{JModeltest} es un programa de \Pname{Java} por lo que funciona de la misma forma en todas las plataformas de computo, la  salida es en modo de texto / tablas dentro del mismo programa; el programa le permite mayor control sobre la forma de iniciar el c\'alculo, tanto desde el \'arbol de inicio hasta la forma de implementar el test jer\'arquico, desde JC hacia GTR (\textit{forward}) o en sentido contrario  (\textit{backward}); 
puede usar hasta 203 modelos y de manera  autom\'atica corre los an\'alisis en m\'ultiples procesadores, lo que hace m\'as eficiente que \Pname{PAUP*} o \Pname{R}, adem\'as puede ejecutarlo desde la l\'inea de comandos con modelos diferenciales lo que facilita su uso en clusters de computadores; \Pname{R} en conjunto con \Rlib{ape} y \Pname{PhyML} no solo puede realizar el c\'alculo del mejor mod\'elo de evoluci\'on molecular sino que puede graficar los resultados. Es posible que obtenga distintos modelos dependiendo de la forma como estructure su an\'alisis. 


\section*{T\'ecnicas}

%\section{Materiales}
%\noindent
%Matrices de datos formato:\\ 
%\Pname{NEXUS} para \Pname{ModelTest} (datos.modelo.dat).\\
%\Pname{PHYLIP} para \Pname{JModelTest} (datos.oualin.dat).\\
%Matriz de instrucciones para R:\\
%\textit{model.R}\\ 
%\section{M\'etodos}
%\subsection{Modeltest+PAUP}
%
\noindent

Par obtener el modelo con programas como \Pname{PAUP*} o \Pname{R} debe tener un archivo de datos con las secuencias alineadas, revise la pr\'actica ~\ref{ch:alinear}. 

%la secuencia b\'asica de instrucciones desde la l\'inea de comandos es:


En \Pname{JModeltest2 + PhyML}


\begin{enumerate}
	\item  Abra en un editor la matriz de datos \Datos{DNA1.phy}. Revise las principales caracter\'isticas del formato \Pname{PHYLIP}

	\item  Abra la matriz de datos en el programa \Pname{JModelTest}.
	
	\item\label{itm:jm2} Calcule los valores de likelihood ({\textit{likelihood scores}}), a partir de un \'arbol base de \Pname{BIONJ}. 
	
	\item Calcule el valor de AIC (criterio de elecci\'on del modelo de mayor ajuste a la matriz). El valor de AIC ser\'a mejor cuanto m\'as \textbf{peque\~no} sea.
	
	\item Revise la salida de AIC y anote cu\'al es el modelo de mayor ajuste para la matriz de datos, as\'i como los par\'ametros de este modelo.
	
	\item Repita el an\'alisis a partir del c\'alculo de criterio bayesiano (BIC). Compare sus resultados, tanto para AIC como para BCI con los obtenidos por sus compa\~neros.

	\item Repita \ref{itm:jm2}  pero use una b\'usqueda de \Pname{FIXED BIONJ+JC} y eval\'ue el resultado del test jer\'arquico. 

	\item Repita desde \ref{itm:jm2} con un \'arbol base de ML optimizado y eval\'ue el modelo.

	\pregunta{?`Existe alguna diferencia entre los modelos sugeridos por cada enfoque? }
\end{enumerate}



En \Pname{R+ape+PhyML}


\begin{enumerate}
	
	\item  Abra \Pname{R} y revise si tiene instalada la biblioteca \Rlib{ape}, si es necesario, instalela.
	
	\item Coloque como directorio activo el directorio donde tenga los datos y \Pname{PhyML}
	
	\item Cree un archivo con las instrucciones\footnote{La instrucci\'on en \Pname{R} para comentarios es $\#$ y por lo tanto no es ubna instrucci\'on del programa}:

\label{sec:phytest}
\Cmdi{
	\begin{itemize}[label=$>$]
  		\item library(ape) $\#$ cargamos la libreria ape
			%\\$\#\#$ el alineamiento es solo por el ejercicio de mostrar
			%\\$\#\#$ lo que debe hacerse y los comandos respectivos
			%\\$\#\#$
		\item DNA $<-$ read.dna($"$alineado.phy$"$) $\#$ leemos datos alineados en formato tipo \Fname{phylip}
		\item table(unlist(lapply(DNA, length)))$\#\#$ listado de tama\~no de las secuencias
		$\#\#$Test de modelos de evoluci\'on con ape/R y modelos via phyml 
		\item library(ape) $\#\#$ cargamos la libreria ape
		\\$\#\#$  con phymltest probamos 28 modelos en phyml
		\\$\#\#$  ape + R hacen todo el proceso
		\\$\#\#$  en Linux cambie a
		\\$\#\#$  execname = $"$./phyml$"$, si es local o
		\\$\#\#$  execname = $"$phyml$"$, si es global
		\\$\#\#$
		\\$\#\#$ en windows use:
		\item modelo $<-$ phymltest($"$alineado.phy$"$,\\ format = $"$sequential$"$, itree = NULL, exclude = NULL,\\ execname = $"$phyml.exe$"$, append = FALSE)
		\\$\#\#$  use format = $"$interleaved$"$ si aplica a su matriz
		\item print(modelo) $\#\#$ con print puede revisar los valores de AIC
		\item summary(modelo) $\#\#$  con summary puede tener un resumen de los valores del test jer\'arquico
		\item plot(modelo, main = $'$test de modelos para ML, usando PhyML$'$)
		$\#\#$ grafica con $'$plot$'$ los valores de AIC
	\end{itemize}
}
%%%%%%%%%


	\item Compare el modelo sugerido por AIC y por el test jer\'arquico.

	\item Cargue cada instrucci\'on por l\'inea de comandos desde un editor de texto y si tiene todas las instrucciones escritas en un archivo \Datos{modelos.R}, puede usar este archivo y ejecutar todas las instrucciones desde la l\'inea de comandos:
	\Cmd{R -f modelos.R} 
	
\end{enumerate}



\\
Tanto \Pname{JModeltest} como \Pname{R+ape} usan \Pname{PhyML} para el c\'alculo de los valores de likelihood; los dos programas permiten evaluar el modelo de una manera m\'as amigable que la l\'inea de comandos y tener o una salida gr\'afica (\Pname{R+ape}) o una salida de texto f\'acilmente le\'ible.




En \Pname{PAUP*}:

\begin{enumerate}		
	\item  Abra la matriz de datos \Datos{DNA1.nex}.
	\item use \Cmd{help automodel;} para obtener ayuda sobre la instrucci\'on  y las opciones en uso.
	\item  construya un \'arbol tal y como lo hizo en la secci\'on \ref{cha:parsimonia},
	use un \'arbol de NJ con distancia de Jukes-Cantor como inicio:
	\Cmdi{\begin{itemize}[label=$>$]
  				\item set criterion=dist;
  				\item dset distance=JC;
  				\item hsearch;
  			\end{itemize}
  			}
	   Este es el \'arbol de inicio para hacer los c\'alculos para la selecci\'on del modelo.
	    
	\item use 
	   \cmd{ automodel}
	   para hacer una b\'usqueda por defecto, use: 
	   \Cmd{ automodel AIC=yes AICc=no BIC=no modelset=j3 lset=AIC invarSites=no gammaRates=no;}
	   para hacer una b\'usqueda con el equivalente a 3 modelos en \Pname{jmodeltest}; compare los resultados con la b\'usqueda anterior.

	\item Calcule el valor de AIC para 4 diferentes modelos: JC, K80, GTR y uno de su elecci\'on. El valor de AIC ser\'a mejor cuanto m\'as \textbf{peque\~no}.

	\item Para los mismos modelos y el escogido por el hLRT en \Pname{JModelTest}, calcule el criterio de informaci\'on bayesiano. Compare sus resultados, tanto para AIC como para BIC con el de sus compa\~neros, y determine cual es o son los mejores modelos para esos criterios. Ordene los modelos del mejor al peor. 

\pregunta{
	\begin{itemize}
		\item ?`Es el modelo escogido por el criterio de Akaike igual al escogido en el test jer\'arquico? ?`Usted esperar\'ia que lo fuesen?
		\item ?`Es igual el resultado en las distintas exploraciones realizadas? ?`Usted esperar\'ia que fuesen iguales \'o desiguales?
		\item ?`Son iguales los resultados en AIC y BIC?
		\item De usar los tres programas ?`usted espera la misma respuesta?
	\end{itemize}
}

\end{enumerate}



\preguntaGral{
	\begin{itemize}
		\item ?`C\'omo se relaciona el concepto de caracteres hom\'ologos usado en los primeros laboratorios, con el concepto de los modelos?\\
		\item ?`Prefer\'ia usted usar siempre el mismo modelo (por ejemplo, JC)? Argumente su respuesta.
	\end{itemize}
}

\section*{Literatura recomendada}


\cite{PosadaCrandall2001} [presentan de manera completa c\'omo seleccionar modelos bas\'andose en la maximizaci\'on del ajuste].

\cite{Swofford1996} [es una referencia b\'asica para comprender el desarrollo de los distintos modelos y los c\'alculos asociados].

