\thispagestyle{empty}
\pagestyle{fancy}
\fancyhead{}
\fancyfoot{}
\renewcommand{\chaptername}{}
\fancyhead[LE]{ \textbf{Sistem\'atica Filogen\'etica. Introducci\'on}}
\fancyhead[RO]{ \textbf{Introducci\'on a la pr\'actica. Segunda Edici\'on}}
\fancyfoot[CE,CO]{\thepage}
\chapter{Introducci\'on general}
Este libro ha sido pensado como uno de los tantos soportes posibles para las clases de sistem\'atica de nivel b\'asico y avanzado, adem\'as de servir de repaso a conceptos te\'oricos generales \textit{pero} sobre la base emp\'irica, y no como \textbf{reemplazo} de los 
libros b\'asicos o avanzados sobre an\'alisis filogen\'etico.
\section*{?`C\'omo est\'a estructurado este manual?}
El libro consta de 12 secciones que van desde manejo de caracteres, pasando por editores de matrices y \'arboles a b\'usquedas tanto para an\'alisis de parsimonia y m\'axima verosimilitud (ML), como para an\'alisis bayesiano. En todos los casos se presentan las t\'ecnicas, la metodolog\'ia a seguir, los programas de c\'omputo a usar (y sus \cmd{comandos}), adem\'as de una serie de preguntas sobre la pr\'actica o en general sobre la t\'ecnica. La literatura recomendada es una sugerencia de lecturas, desde el punto de vista de los autores, cr\'iticas, pero obviamente no cubre todos los art\'iculos posibles; exploraciones constantes de revistas como \textit{Cladistics}, \textit{Systematic Biology}, \textit{Zoologica Scripta} o \textit{Molecular Phylogenetics and Evolution} y similares, actualizar\'ian las perspectivas aqu\'i presentadas. Al final se presenta un cap\'itulo que trata sobre los programas usados, que se espera funcione como una guia r\'apida para el uso de los mismos pero que no reemplaza al manual.\\
A lo largo del libro se utiliza una tipograf\'ia consistente para indicar el nombre de los programas (v.g., \Pname{TNT}, \Pname{Component}), las instrucciones que deben ser escritas en los programas de l\'inea de comando, por ejemplo: \Cmd{mult=replic 10;} 
y las instrucciones que se acceden mediante un men\'u o un cuadro de dialogo en los programas de interfaz gr\'afica (v.g., \Gui{Analize/heuristics}).\\

El orden de los cap\'itulos obedece a la estrategia de ense\~nanza de la UIS pero  el manual puede ser seguido en otros \'ordenes, por ejemplo todo sobre caracteres, excluyendo alineamiento, seguido de b\'usquedas incluida la b\'usqueda del modelo, consensos y por \'ultimo soporte y al final discutir alineamiento.\\ 
Usted podr\'a encontrar material adicional, los datos, algunos macros para los distintos programas y dem\'as chismes en el sitio web del laboratorio de Sistem\'atica \& Biogeograf\'ia (LSB) de la UIS, en la direcci\'on: http://tux.uis.edu.co/labsist.
% \href{http://tux.uis.edu.co/labsist/docencia}.
\section*{P\'ublico objeto}
Se espera que el usuario de este manual tenga conocimientos b\'asicos, o que est\'e tomando un curso formal de sistem\'atica filogen\'etica a nivel de pre o posgrado. No se esperan caracter\'isticas especiales en cuanto a dominio de computadores, pero el manejo de un editor de texto que permita grabar archivos sin formato es \emph{muy} recomendable, adem\'as de manejar el entorno de l\'inea de comandos, el cual es usado en una gran cantidad de programas. Los usuarios pueden trabajar en cualquier plataforma desde Linux a MacOSX, pasando por Windows; en general se citan los programas apropiados para cada plataforma.


\section*{Agradecimientos}
A la Universidad Industrial de Santander, en particular a la Escuela de Biolog\'ia. A los estudiantes del programa de pregrado de la escuela de Biolog\'ia de la UIS (Universidad Industrial de Santander) 
estudiantes profundizaci\'on

a Viviana por reescribir el cap\'itulo de b\'usquedas



.