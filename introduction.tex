\chapter{Introduction}

\section{Rhizobia} \label{s-rhizobia-intro}

In this thesis `rhizobia' are defined as bacteria capable of forming
root nodules on legumes, mediated by \emph{nod} genes. This term
describes the phenotype (causing root nodules), but has no taxonomic
relevance and should not be confused with the genus name
\emph{Rhizobium} (although `rhizobia' has been used by others for
the plural form of \emph{Rhizobium}). An equivalent term used by
other researchers is `root nodule bacteria' (RNB)
\citep{Zakhia04,Howieson05}.

Rhizobia are soil-inhabiting bacteria with the potential for forming
specific root structures called nodules. In effective nodules the
bacteria fix nitrogen gas (N$_2$) from the atmosphere into ammonia
\citep{OGara76}, which is assimilated by the plant and supports
growth---particularly in nutrient deficient soils. In return the
rhizobia are supplied with nutrients (predominantly dicarboxylic
acids \citep{Lodwig03}), and are protected inside the nodule
structure \citep{VanRhijn95}. In ineffective nodules no nitrogen is
fixed, yet rhizobia are still supplied with nutrients, and in this
situation the rhizobia could be considered parasitic
\citep{Denison04b}.

The nitrogen-fixing symbiotic relationship has been exploited in
agriculture to enhance crop and pasture growth without the addition
of nitrogen fertilisers. For this reason, the majority of research
in this field has focused on herbaceous crop and forage legumes of
agricultural significance. In contrast, few studies have been made
of rhizobial associations among non-crop legumes, despite the fact
that they may be ecologically important in the natural landscape
\citep{Boring88}.

Worldwide, there are an estimated 17\,000--19\,000 legume species
\citep{Martinez96}, although nodulating bacterial species have only
been identified for a small proportion of these. To date (September
2006), 55 rhizobial species have been identified in twelve genera
(Table \ref{t-Rhizobia-species}). Most of the species are in the
genera \emph{Rhizobium} (from the Latin `root living'),
\emph{Bradyrhizobium}, \emph{Mesorhizobium} and \emph{Ensifer}
(\emph{Sinorhizobium}).

A detailed discussion of rhizobial systematics is presented in
Section \ref{s-systematics}, but important taxonomic distinctions
are noted below. All currently known rhizobia are in the phylum
Proteobacteria, most in the class Alphaproteobacteria, which
contains six rhizobial families in a single order---Rhizobiales, as
listed in the hierarchy below \citep{Bergeys-outline}.

\singlespacing

\begin{tabbing}
  0000 \= 0000 \= 0000 \= 0000 \kill
  % \> for next tab, \\ for new line...
  Rhizobiales \\
  \> Rhizobiaceae  \\
  \> \> \emph{Rhizobium} \\
  \> \> \emph{Ensifer} (\emph{Sinorhizobium}) \\
  \> Brucellaceae  \\
  \> \> \emph{Ochrobactrum} \\
  \> Phyllobacteriaceae  \\
  \> \> \emph{Phyllobacterium} \\
  \> \> \emph{Mesorhizobium} \\
  \> Bradyrhizobiaceae  \\
  \> \> \emph{Bradyrhizobium} \\
  \> Hyphomicrobiaceae  \\
  \> \> \emph{Azorhizobium} \\
  \> \> \emph{Devosia} \\
  \> Methylobacteriaceae  \\
  \> \> \emph{Methylobacterium} \\
\end{tabbing}

\onehalfspacing
%\doublespacing

There are also three rhizobial species in two families in the
Betaproteobacteria, all of which are in the Burkholderiales order,
as listed below \citep{Bergeys-outline}.

\singlespacing

\begin{tabbing}
  0000 \= 0000 \= 0000 \= 0000 \kill
  % \> for next tab, \\ for new line...
  Burkholderiales \\
  \> Burkholderiaceae  \\
  \> \> \emph{Burkholderia} \\
  \> \> \emph{Cupriavidus} \\
  \> Oxalobacteraceae  \\
  \> \> \emph{Herbaspirillum} \\
\end{tabbing}

\onehalfspacing
%\doublespacing


Although it remains to be confirmed, it is possible that some
Gammaproteobacteria also nodulate legumes \citep{Benhizia04}.

%\vspace{2cm}

\singlespacing

% \afterpage{\clearpage \input{rhizobia-sp.tex}}
  {\small
 \begin{longtable}{ll}

   \caption{List of rhizobial species} \\

    \toprule
    \textbf{Binomial Name} & \textbf{Authority}$^a$ \\
    \midrule
    \emph{Rhizobium daejeonense}                    & \citealp{Quan05}                      \\
    \emph{Rhizobium etli}                           & \citealp{Segovia93}                   \\
    \emph{Rhizobium galegae}                        & \citealp{Lindstrom89}                 \\
    \emph{Rhizobium gallicum}                       & \citealp{Amarger97}                   \\
    \emph{Rhizobium giardinii}                      & \citealp{Amarger97}                   \\
    \emph{Rhizobium hainanense}                     & \citealp{Chen97}                      \\
    \emph{Rhizobium huautlense}                     & \citealp{Wang98}                      \\
    \emph{Rhizobium indigoferae}                    & \citealp{Wei02}                       \\
    \emph{Rhizobium leguminosarum}$^{\mathrm{T}}$   & \citep{Frank79} \citealp{Frank89}     \\
    \emph{Rhizobium loessense}                      & \citealp{Wei03}                       \\
    \emph{Rhizobium mongolense}                     & \citealp{vanBerkum98}                 \\
    \emph{Rhizobium sullae}                         & \citealp{Squartini02}                 \\
    \emph{Rhizobium tropici}                        & \citealp{Martinez91}                  \\
    \emph{Rhizobium undicola}                       & \citep{deLajudie98a} \citealp{Young01a} \\
    \emph{Rhizobium yanglingense}                   & \citealp{Tan01a}                       \\
    \cmidrule{1-2}
    \emph{Ensifer} (\emph{Sinorhizobium}) \emph{abri}                       & \citealp{Ogasawara03} \\
    \emph{Ensifer adhaerens}                  & \citep{Wang02a} \citealp{Young03b}\\
    \emph{Ensifer} (\emph{Sinorhizobium}) \emph{americanum}                 & \citealp{Toledo03}                        \\
    \emph{Ensifer arboris}                    & \citep{Nick99} \citealp{Young03b}                         \\
    \emph{Ensifer fredii}$^{\mathrm{T}}$      & \citep{Scholla84} \citealp{Young03b}      \\
    \emph{Ensifer} (\emph{Sinorhizobium}) \emph{indiaense}                  & \citealp{Ogasawara03} \\
    \emph{Ensifer kostiensis}                  & \citep{Nick99}  \citealp{Young03b}                        \\
    \emph{Ensifer kummerowiae}                & \citep{Wei02} \citealp{Young03b}                          \\
    \emph{Ensifer medicae}                    & \citep{Rome96}  \citealp{Young03b}                        \\
    \emph{Ensifer meliloti}                   & \citep{Dangeard26} \citealp{Young03b}     \\
    \emph{Ensifer saheli}                     & \citep{deLajudie94}  \citealp{Young03b}                   \\
    \emph{Ensifer terangae}                   & \citep{deLajudie94} \citealp{Young03b}                    \\
    \emph{Ensifer xinjiangense}               & \citep{Chen88}  \citealp{Young03b}                        \\
    \cmidrule{1-2}
    \emph{Mesorhizobium amorphae}               & \citealp{Wang99a}                     \\
    \emph{Mesorhizobium chacoense}              & \citealp{Velazquez01a}                \\
    \emph{Mesorhizobium ciceri}                 & \citep{Nour94a} \citealp{Jarvis97}    \\
    \emph{Mesorhizobium huakuii}                & \citep{Chen91} \citealp{Jarvis97}     \\
    \emph{Mesorhizobium loti}$^{\mathrm{T}}$    & \citep{Jarvis82} \citealp{Jarvis97}   \\
    \emph{Mesorhizobium mediterraneum}          & \citep{Nour95} \citealp{Jarvis97}     \\
    \emph{Mesorhizobium plurifarium}            & \citealp{deLajudie98b}                \\
    \emph{Mesorhizobium septentrionale}         & \citealp{Gao04b}                      \\
    \emph{Mesorhizobium temperatum}             & \citealp{Gao04b}                      \\
    \emph{Mesorhizobium tianshanense}           & \citep{Chen95} \citealp{Jarvis97} \\
    \cmidrule{1-2}
    \emph{Bradyrhizobium canariense}                & \citealp{Vinuesa05b}                       \\
    \emph{Bradyrhizobium elkanii}                   & \citealp{Kuykendall93}                    \\
    \emph{Bradyrhizobium japonicum}$^{\mathrm{T}}$  & \citep{Kirchner96} \citealp{Jordan82}     \\
    \emph{Bradyrhizobium liaoningense}              & \citealp{Xu95}                            \\
    \emph{Bradyrhizobium yuanmingense}              & \citealp{Yao02}                           \\
    \cmidrule{1-2}
    \emph{Burkholderia caribensis}           & \citealp{Vandamme02} \\
    \emph{Burkholderia cepacia}              & \citealp{Vandamme02} \\
    \emph{Burkholderia phymatum}             & \citealp{Vandamme02} \\
    \emph{Burkholderia tuberum}              & \citealp{Vandamme02} \\
    \cmidrule{1-2}
    \emph{Azorhizobium caulinodans}$^{\mathrm{T}}$  & \citealp{Dreyfus88} \\
    \emph{Azorhizobium doebereinerae}               & \citealp{Souza06} \\
    \cmidrule{1-2}
    \emph{Cupriavidus taiwanensis}                  & {\footnotesize \citep{Chen01x} \citealp{Vandamme04}} \\
    \cmidrule{1-2}
    \emph{Devosia neptuniae}                        & \citealp{Rivas03}  \\
    \cmidrule{1-2}
    \emph{Herbaspirillum lusitanum}                 & \citealp{Valverde03} \\
    \cmidrule{1-2}
    \emph{Phyllobacterium trifolii}                 & \citealp{Valverde03} \\
    \cmidrule{1-2}
    \emph{Methylobacterium nodulans}                & \citealp{Jourand04} \\
    \cmidrule{1-2}
    \emph{Ochrobactrum lupini}                      & \citealp{Trujillo05} \\
    \bottomrule

  \label{t-Rhizobia-species}

  \end{longtable} }

  \vspace{-0.9cm}
  {\footnotesize
  \noindent $^{\mathrm{T}}$ Type species \\
  $^a$ Parenthesis indicate original publication, following reference is subsequent
  reclassification \\ }
  \medskip

\onehalfspacing

There are a number of species present in these rhizobial genera that
have not been observed to form nodules, and therefore do not fit the
functional definition of rhizobia. These include many of the species
that were formerly known as \emph{Agrobacterium} (e.g.
\emph{R.~larrymoorei}, \emph{R.~rubi}, and \emph{R.~vitis};
\citep{Young01a,Young04a}). However, there is recent evidence that
other species formerly classified as \emph{Agrobacterium} are
capable of nodulation. For example \emph{R.~radiobacter}\footnote{As
\emph{Agrobacterium tumefaciens} in the publication.} nodulates
\emph{Phaseolus vulgaris}, \emph{Campylotropis} spp., \emph{Cassia}
spp.~\citep{Han05}, and \emph{Wisteria sinensis} \citep{Lui05}. Both
nodules and tumours were formed on \emph{Phaseolus vulgaris} by
\emph{R.~rhizogenes} strains containing a Sym plasmid
\citep{Velazquez05}.

There are also other species, although classified within genera
commonly considered to be represented entirely by nodulating
strains, in fact include strains apparently devoid of nodulation
ability. For example \emph{Bradyrhizobium betae} forms tumours on
\emph{Beta vulgaris} (Beetroot) but is not known to fix N$_2$
\citep{Rivas04a}. \emph{Mesorhizobium thiogangeticum} is a
sulfur-oxidising bacterium, and does not nodulate the tested legumes
of \emph{Clitoria ternatea}, \emph{Pisum sativum}, and \emph{Cicer
arietinum} \citep{Gosh06}. There are also non-symbiotic strains of
\emph{Mesorhizobium} (and other genera) that can become nodulating
species by acquiring symbiosis genes \citep{Sullivan95}.

The genus \emph{Sinorhizobium} was recently reclassified to
\emph{Ensifer} on the basis of similarity of DNA sequences and
priority of publication \citep{Willems03,Young03b}. \emph{Ensifer
adhaerens} is a soil bacterium that attaches to other bacteria and
may cause cell lysis \citep{Casida82}. Although wild type
\emph{E.~adhaerens} did not nodulate \emph{Phaseolus vulgaris} nor
\emph{Leucaena leucocephala}, it did so when transformed with a
symbiotic plasmid from \emph{Rhizobium tropici} \citep{Rogel01},
demonstrating its capacity to become a rhizobial species. Other
\emph{E.~adhaerens} strains were subsequently isolated that
nodulated legumes naturally. These form a single clade with
\emph{Sinorhizobium} in 16S rRNA and \emph{recA} phylogenies leading
\citet{Willems03} to suggest that these strains be reclassified as
\emph{Sinorhizobium adhaerens}. However, \emph{Ensifer}
\citep{Casida82} is the senior heterotypic synonym and thus takes
priority \citep{Young03b}. This means that all \emph{Sinorhizobium}
spp.~must be renamed as \emph{Ensifer} spp.~according to the
Bacteriological code \citep{Lapage90}. In this thesis \emph{Ensifer}
is used exclusively.

\emph{Cupriavidus} species have recently undergone several taxonomic
revisions, being formerly known as both \emph{Wautersia} and
\emph{Ralstonia}. This genus currently contains a single rhizobial
species, and ten other non-symbiotic species \citep{Euzeby97}.

Rhizobial systematics is rapidly changing, and  recently many new
species have been recognised. Novel species may also be associated
with the native legumes of New Zealand.




%--------------------------------------------------------------------------------

\section{New Zealand native legumes} \label{s-NZ-natives}

\subsection{Introduction}

New Zealand has 33 species of legumes that are native. These are
comprised of four genera: \emph{Sophora}, \emph{Carmichaelia},
\emph{Clianthus}, and \emph{Montigena}.



%------------------------------------------------------------------------------------
\subsection{\emph{Sophora}}

\begin{figure} [tb]
    \centering
    \includegraphics[width=12cm]{Sophora}
    \caption[\emph{Sophora chathamica}]{\emph{Sophora chathamica}
     showing yellow bell-shaped flowers and mature seed pod (right of centre).}
    \label{p-Sophora}
\end{figure}

\emph{Sophora} L.~(1753) was named after \emph{sufayra}, the arabic
name for the tree. The M\=aori name (and the vernacular) for the
endemic \emph{Sophora} is `k\=owhai', from the word for
yellow---which describes the colour of the flowers
(Fig.~\ref{p-Sophora}).

There are eight species native to New Zealand: \emph{S.~chathamica},
\emph{S.~fulvida}, \emph{S.~godley}, \emph{S.~longicarinata},
\emph{S.~microphylla}, \emph{S.~molloyi}, \emph{S.~prostrata} and
\emph{S.~tetraptera} \citep{Heenan01c}. There are another 49 species
in  the genus \emph{Sophora} that are not native to New Zealand.
Species endemic to the Southern Hemisphere are in the
\emph{Edwardsia} sector of \emph{Sophora}. \emph{Edwardsia} members
other than the New Zealand natives are from South America
(\emph{S.~macrocarpa}), Lord Howe Island (\emph{S.~howinsula}),
Hawaii (\emph{S.~chrysophylla}), La R\'eunion (\emph{S.~denudata}),
Easter Island (\emph{S.~toromiro}), and Raivavae Island
(\emph{S.~raivavae}). \emph{S.~microphylla} was considered to occur
in Chile and on Gough Island in the south Atlantic
\citep{Markham72}, however \citet{Heenan01a} considers these species
to be \emph{Sophora cassioides}, distinct from the New Zealand
species. The type species of the genus is
\emph{S.~tomentosa}$^{\mathrm{T}}$ which is closely related to
sect.~\emph{Edwardsia} \citep{Heenan04,ILDIS}.

%------------------------------------------------------------------------------------
\subsection{\emph{Carmichaelia}}

\begin{figure} [tb]
    \centering
    \includegraphics[width=12cm]{Carmichaelia}
    \caption[\emph{Carmichaelia australis}]{\emph{Carmichaelia australis}
    showing the cladodes and mature seed pods. Insert: 2$\times$ magnification
    of flowers from the same plant.}
    \label{p-Carmichaelia}
\end{figure}

\emph{Carmichaelia} R.Br.~(1825) was named after Captain Dugald
Carmichael, a Scottish army officer and botanist who collected
plants in New Zealand \citep{Allen81}. The English vernacular name
is `New Zealand broom', and in M\=aori is variably known  as tawao,
m\=akaka, maukoro, and tainoka \citep{NZPlant} (illustrated in
Fig.~\ref{p-Carmichaelia}).

The taxonomic history of this genus is complex, and has been
confused by inadequate collections and intraspecific variation
\citep{Heenan95b}. The formerly recognised genera of
\emph{Chordospartium}, \emph{Corallospartium}, \emph{Notospartium},
and \emph{Huttonella} are now included in \emph{Carmichaelia}
\citep{Heenan95a,Heenan98c,Heenan98a}. In the most recent treatment
\citep{Heenan95b,Heenan96b}, there are 22 species of
\emph{Carmichaelia} native to New Zealand (\emph{C.~appressa},
\emph{C.~arborea}, \emph{C.~astonii},
\emph{C.~australis}$^{\mathrm{T}}$, \emph{C.~carmichaeliae},
\emph{C.~compacta}, \emph{C.~corrugata}, \emph{C.~crassicaule},
\emph{C.~curta}, \emph{C.~glabrescens}, \emph{C.~hollowayi},
\emph{C.~juncea}, \emph{C.~kirkii}, \emph{C.~monroi},
\emph{C.~muritai}, \emph{C.~nana}, \emph{C.~odorata},
\emph{C.~petriei}, \emph{C.~stevensonii}, \emph{C.~torulosa},
\emph{C.~uniflora}, \emph{C.~vexillata}, and \emph{C.~williamsii}).
An additional species, \emph{C.~exsul}, is found on Lord Howe Island
in the Tasman Sea, 600 km east from Australia. The species exhibit
remarkable diversity, from trees to prostrate forms a few
centimetres high.

\emph{Carmichaelia} is distributed throughout New Zealand, although
most species are restricted to certain localities. Most of the
diversity (15 species) is in the eastern South Island. They
typically invade disturbed habitats on shallow poor soils, drought
and frost prone areas, and alluvial soils \citep{Wagstaff99}.

%------------------------------------------------------------------------------------
\subsection{\emph{Clianthus}}

\begin{figure} [tb]
    \centering
    \includegraphics[width=12cm]{Clianthus}
    \caption[\emph{Clianthus puniceus}]{\emph{Clianthus puniceus}
     showing distinctive beak-shaped flowers.}
    \label{p-Clianthus}
\end{figure}


\emph{Clianthus} Soland. ex Lindl. was named from the Greek
\emph{kleos} `glory' and \emph{anthos} `flower' \citep{Allen81}. The
English vernacular name is `kakabeak' after its distinctive flowers
shaped like a native parrot's (k\=ak\=a) beak
(Fig.~\ref{p-Clianthus}), it is known in M\=aori as `k\=owhai
ngutuk\=ak\=a' \citep{Shaw97}.

Once considered monotypic, in the most recent treatment
\citep{Heenan95c,Heenan00}, there are now two species
(\emph{C.~maximus} and \emph{C.~puniceus}$^{\mathrm{T}}$) native to
New Zealand. It is found naturally only in isolated refuges in the
eastern North Island. Formerly some Australian and Asian legumes
were classified as \emph{Clianthus}, these are now known as
\emph{Swainsona} and \emph{Sarcodum} \citep{ILDIS}.




%----------------------------------------------------------------------------

\subsection{\emph{Montigena}}

\begin{figure} [tb]
    \centering
    \includegraphics[width=12cm]{Montigena}
    \caption[\emph{Montigena novae-zelandiae}]{\emph{Montigena
    novae-zelandiae}, growing on a scree slope,
     with mature seed pods. Photo \copyright\ Peter Heenan.}
    \label{p-Montigena}
\end{figure}

\emph{Montigena} (Hook.f.) Heenan, is named from `mountain-born'
referring to its habitat. \citep{Heenan98d}. The English vernacular
name is `scree pea' (Fig.~\ref{p-Montigena}).

\emph{Montigena novae-zelandiae}$^{\mathrm{T}}$ is the only species
in the \emph{Montigena} genus. It was known as \emph{Swainsona
novae-zelandiae} until \citet{Heenan98d} reclassified it based on
morphological features. There are currently 55 \emph{Swainsona}
species, mostly in Australia \citep{ILDIS}. \emph{Montigena} has a
distinctly different ITS sequence from other New Zealand legumes,
but forms a clade with the Australian \emph{Swainsona}
\citep{Wagstaff99} (See Fig.~\ref{p-legume-tree}).

\emph{Montigena} is endemic to the dry eastern mountains of the
South Island of New Zealand, where it grows on partially stable
scree slopes.



%----------------History---------------------------------------

\section{Evolution and history of New Zealand native legumes} \label{legume-history}

\subsection{Geology and palaeobotany}

The archipelago of New Zealand began to split away from the larger
landmass of Gondwana about 80 million years ago (mya) due to
continental drift, although was still relatively close for another
10 to 20 million years \citep{Cooper93,Stevens95}. The start of this
separation coincides approximately with the date of the evolution of
legumes \citep{Sprent94}, although legumes were not abundant until
35--54 million years ago  \citep{Doyle03}. Hence all legumes must
have arrived in New Zealand after its separation from Gondwana.

The historical presence of legumes in New Zealand is largely
inferred from fossil pollen records. Fossils of \emph{Carmichaelia}
were detected in  the ``late Pliocene Waipaoa series''of soils
dating from less than 5 mya \citep{Oliver28}, but \emph{Sophora} is
not common in fossil pollen records until the Pleistocene (1.81 mya)
\citep{Hurr99}. Fossil pollen records also show that before the last
Ice Age ended, 10\,000 years ago, New Zealand had an indigenous
population of \emph{Acacia} spp.~\citep{Mildenhall72,Lee01}.

\begin{table} [tbp]
\caption{Native legume taxonomic hierarchy}
  \centering
  \begin{tabular}{|c|c|c|c|c|}
    \hline \textbf{Kingdom}     & \multicolumn{4}{c|}{Plantae} \\
    \hline \textbf{Division}    & \multicolumn{4}{c|}{Magnoliophyta} \\
    \hline \textbf{Class}       & \multicolumn{4}{c|}{Magnoliopsida} \\
    \hline \textbf{Order}       & \multicolumn{4}{c|}{Fabales} \\
    \hline \textbf{Family}      & \multicolumn{4}{c|}{Fabaceae} \\
    \hline \textbf{Subfamily}   & \multicolumn{4}{c|}{Faboideae} \\
    \hline \textbf{Tribe}       & Sophoreae         & \multicolumn{2}{c|}{Carmichaeliaeae$^a$} & Galegeae$^a$    \\
    \hline \textbf{SubTribe}    &                   & \multicolumn{3}{c|}{Carmichaelinae$^b$}              \\
    \hline \textbf{Genus}       & \emph{Sophora}    & \emph{Carmichaelia} & \emph{Montigena} & \emph{Clianthus}  \\
    \hline
  \end{tabular}

   \label{taxbox-Native}
  \medskip
  \raggedright
  {\footnotesize
\hskip 1cm $^a$ After \citet{Pohill81-Carm}. \\
\hskip 1cm $^b$ After \citet{Wagstaff99}. \\}
\end{table}

\subsection{The Carmichaelinae}

The original classification of native legumes placed
\emph{Carmichaelia} and \emph{Montigena} in the tribe
Carmichaelieae, and \emph{Clianthus} in the diverse tribe Galegeae
\citep{Pohill81-Carm}; however this classification is polyphyletic
\citep{Wagstaff99}, and recent evidence has suggested that
\emph{Carmichaelia}, \emph{Clianthus}, \emph{Montigena}, and the
Australian genus \emph{Swainsona}, form a single clade called
Carmichaelinae at the subtribe rank \citep{Wagstaff99}
(Table~\ref{taxbox-Native}).

\citet{Wagstaff99} used ITS sequences of 39 species of
\emph{Carmichaelia}, \emph{Clianthus}, \emph{Montigena},
\emph{Swainsona} and related legumes, to determine the
classification and origins of New Zealand legumes. Most species of
\emph{Carmichaelia} had nearly identical ITS sequences, indicating
recent radiation. The results suggested that Carmichaelinae were
derived from the Northern Hemisphere Astragalinae, and confirmed an
earlier study of \citet{Heenan98a} using 47 phenotypic characters.

\begin{figure} [tbp]
    \centering
    \includegraphics[width=12cm]{legume-tree}
    \caption[Carmichaelinae phylogenetic tree]{Phylogenetic tree of the New Zealand
    Carmichaelinae clade (in bold) and related genera (Galegeae tribe) from Australia
    and other countries, using ITS sequences. This
    penalised likelihood rate-smoothed Bayesian consensus phylogeny and
    the estimated ages were derived from the data and information
    provided in \citet{Wagstaff99}. Carmichaelinae are mainly of
    Australia but with two independent lines in New Zealand. Figure
    modified from \citet{Lavin04}, misspelling of `\emph{Swainsona}' in original. Symbols: * -- \emph{Clianthus puniceus.}}
    \label{p-legume-tree}
\end{figure}

\citet{Lavin04} extended the work of \citet{Wagstaff99} by
reanalysing the data using Bayesian methods to estimate the age of
divergence of each clade (Fig.~\ref{p-legume-tree}). From these data
it appears that the New Zealand Carmichaelinae, including all
\emph{Carmichaelia} species and \emph{Clianthus} (marked on the tree
by `*') diverged 5.3$\pm$1.1 mya, and all Carmichaelinae have a
common origin 7.5$\pm$0.8 mya. \emph{Carmichaelia} shares a common
ancestor with \emph{Sutherlandia} (found in Australia, Africa, and
Mauritius \citep{ILDIS}) 10.4$\pm$2.0 mya. These dates agree with
those from fossil pollen.

In a large study of 235 genera using \emph{matK} sequence data,
\citet{Wojciechowski04} included \emph{Clianthus} and
\emph{Carmichaelia} in a larger ``Astragalean clade'' including
\emph{Swainsona}, \emph{Colutea}, \emph{Sutherlandia},
\emph{Oxytropis}, and \emph{Astragalus}.

These publications show that the radiation of Carmichaelinae legume
species into New Zealand was quite recent (compared to the
diversification of legumes in the Northern Hemisphere). The ancestor
of the Carmichaelinae derived from a Northern Hemisphere lineage and
arrived (probably in Australia) between 10 and 7.5 mya.


\subsection{New Zealand \emph{Sophora}}

\emph{Sophora} is distinct from the other legume genera of New
Zealand, being a member of the Sophoreae tribe (Table
\ref{taxbox-Native}). \emph{Sophora} is a diverse genus that has
about 80 members spread throughout the world. Molecular analyses
indicate that the genus is polyphyletic, and and comprises three
distinct and unrelated lineages
\citep{Kass95,Kass96,Crisp00,Pennington01}.

New Zealand \emph{Sophora} belong to a subset known as
``\emph{Sophora} sect.~\emph{Edwardsia}'' \citep{Kass97,Heenan04}.
This sector is one of the largest groups in \emph{Sophora}, and
includes about 19 species whose distribution is centred on islands
in the southern Pacific Ocean. Most species of
sect.~\emph{Edwardsia} have identical  ITS sequences, indicating a
recent and rapid radiation \citep{Mitchell02}.

There are competing theories on the origin of \emph{Sophora}
sect.~\emph{Edwardsia}. Some believe that they originated in Chile
from a North American ancestor \citep{Sykes68,Pena00}. Molecular
genetics indicates the likely origin is from the North Western
Pacific, from an Eurasian ancestor, in the last 2--5 million years,
and dispersal occurred around the pacific \emph{via} the buoyant
saltwater-resistant seeds \citep{Hurr99,Mitchell02,Heenan04}.

In summary it is proposed that New Zealand \emph{Sophora}
spp.~derived from a separate legume lineage and geographical origin
than the Carmichaelinae, and were dispersed to New Zealand perhaps a
few million years later.




\section{Exotic weed legumes in New Zealand}

\subsection{Introduction}

The indigenous people of New Zealand---the M\=aori---arrived in the
mid 13th century from Eastern Polynesia \citep{Irwin05}. They
brought with them new species, such as mammals (rats, dogs) and
tuber plants (k\=umera, taro, yam), but there is no evidence that
they brought any legumes \citep{Bellich96}.

The first exotic legumes were introduced into New Zealand by
settlers from Europe in the early 19th century. The settlers brought
many plants and animals to establish familiar industries, and to
remind them of their previous homelands. In their endemic habitats,
these shrubs are in equilibrium with their natural flora, but in New
Zealand, some have become serious invasive noxious weeds.

Legumes have several properties that make them successful invaders.
They have a high seed production, often with many seeds per pod, and
many pods per tree. Most legume seeds are able to survive long
periods in soil banks due to their thick impervious testa
\citep{Lee01}. The mature plant generally lives for many years, and
high-density seedling success allows rapid coverage of large areas.
Possibly the success of legumes as invasive weeds is augmented by
their ability to grow in nutrient deficient soils, in association
with nitrogen-fixing rhizobia.


There are now over 100 naturalised legume species in New Zealand
\citep{NZPlant}. A small number of these have become common weeds
and include: \emph{Chamaecytisus palmensis} (tree lucerne),
\emph{Cytisus scoparius} (broom), \emph{Galega officinalis} (goat's
rue), \emph{Lathyrus latifolius} (everlasting pea), \emph{Lotus
pedunculatus} (lotus), \emph{Lotus suaveolens} (hairy birdsfoot
trefoil), \emph{lupinus arboreus} (tree lupin), \emph{Medicago
lupulina} (black medick), \emph{Medicago sativa} (lucerne),
\emph{Melilotus indicus} (King Island melilot), \emph{Ornithopus
perpusillus} (wild serradella), various wattles (\emph{Acacia}
spp.), \emph{Psoralea pinnata} (dally pine), various
\emph{Trifolium} spp.~(clover), \emph{Ulex europaeus} (gorse),
\emph{Vicia hirsuta} (hairy vetch), \emph{Vicia sativa} (vetch)
\citep{Roy04}. The woody species of \emph{Ulex}, \emph{Cytisus}, and
\emph{Acacia} are the most invasive, and are the three of this
study.



\begin{table}
\caption{Weed legume taxonomic hierarchy}
  \centering
  \begin{tabular}{|c|c|c|c|}
    \hline \textbf{Kingdom}    & \multicolumn{3}{c|}{Plantae} \\
    \hline \textbf{Division}   & \multicolumn{3}{c|}{Magnoliophyta} \\
    \hline \textbf{Class}      & \multicolumn{3}{c|}{Magnoliopsida} \\
    \hline \textbf{Order}      & \multicolumn{3}{c|}{Fabales} \\
    \hline \textbf{Family}     & \multicolumn{3}{c|}{Fabaceae} \\
    \hline \textbf{Subfamily}  & \multicolumn{2}{c|}{Faboideae}    & Mimosoideae\\
    \hline \textbf{Tribe}      & \multicolumn{2}{c|}{Genisteae}     & Acacieae\\
    \hline \textbf{Genus}      & \emph{Ulex} & \emph{Cytisus}      & \emph{Acacia} \\
    \hline
  \end{tabular}

   \label{taxbox-Weed}
  \medskip
  \raggedright
  {\footnotesize
\hskip 3cm Note: Information from \citet{ILDIS}}
\end{table}


\emph{Ulex} and \emph{Cytisus} are classified in the Genisteae
tribe, and \emph{Acacia} is in the Acacieae tribe
(Table~\ref{taxbox-Weed}). These are distinct from the woody native
New Zealand legumes, which belong to the tribes Sophoreae and
Carmichaelinae.





%--------------------------------------------------------------------------------

\subsection{\emph{Ulex europaeus}}

\begin{figure} [tb]
    \centering
    \includegraphics[width=12cm]{Ulex}
    \caption[\emph{Ulex europaeus}]{\emph{Ulex europaeus} showing
     spines and flowers beginning to open.}
    \label{p-Ulex}
\end{figure}



\emph{Ulex europaeus} L.~is known in the vernacular as whin, furze,
or more commonly in New Zealand---gorse (Fig~\ref{p-Ulex}).

There are some eleven \emph{Ulex} species but only
\emph{U.~europaeus} is important in New Zealand \citep{ILDIS}. Its
habitat is mostly disturbed and modified ecosystems, including
river-beds, pasture, scrubland, forest margins and wasteland.

Gorse is native to Western Europe and was naturalised in New Zealand
in 1867 \citep{Bellingham04}, although \citeauthor{Darwin09}
recorded it at Waimate some thirty years earlier in December 1835.
It was introduced as a `living fence', but outgrew its usefulness
and was soon classified as a weed. It is now considered New
Zealand's worst weed, and millions of dollars are spent annually in
control \citep{Hill86}. Gorse is also a problem in parts of Spain,
Portugal, Chile, Hawaii, Ireland, coastal Oregon, and Southern
Australia \citep{Roy04,Leary06}

%Recently however some biocontrol agents, the gorse spider mite and
%... ask Chris. Gorse in some situations may be beneficial in the
%long term (25+ years) as mature gorse strands thin out and die
%allowing other vegetation to grow. However burning resets this
%cycle.

%--------------------------------------------------------------------------------

\subsection{\emph{Cytisus scoparius}}


\begin{figure} [tb]
    \centering
    \includegraphics[width=12cm]{Cytisus}
    \caption[\emph{Cytisus scoparius}]{\emph{Cytisus scoparius}. Photo \cc Jon J. Sullivan.}
    \label{p-Cytisus}
\end{figure}

\emph{Cytisus scoparius}~(L.)~Link, is also classified as
\emph{Sarothamnus scoparius} (L.) W.D.J. Koch\footnote{Occasionally
incorrectly spelt as \emph{Sarathamnus}.}. It is commonly called
`broom' or `scotch broom'. \emph{Cytisus} has some 51 taxa
\citep{ILDIS}, but only \emph{C.~scoparius} is of importance in New
Zealand (Fig.~\ref{p-Cytisus}).

Broom is common throughout New Zealand, particularly on the drier
eastern side of the South Island, and the central North Island
\citep{Fowler00}. Its habitat is mostly river-beds, hedgerows,
low-fertility hill country, scrubland, coastal areas, and waste
land. It was originally from Europe, Asia, and Russia. In New
Zealand it grows more vigorously than in its native range, with a
greater maximum age and larger size. It was naturalised in New
Zealand in 1872 \citep{Bellingham04}.

Broom causes economic losses to agricultural and forestry
operations, and occupies 0.92\% of South Island farmable land
\citep{Fowler00}.

%--------------------------------------------------------------------------------

\subsection{\emph{Acacia}}

\emph{Acacia} (commonly called wattle) is a  large genus with over
950 species \citep{ILDIS}. Recent studies have shown that
\emph{Acacia} is polyphyletic and should be split into five genera
\citep{Luckow05}, although there are competing proposals for this.
`Proposal 1584' would retypify \emph{Acacia}: The type of the
Australian taxon (\emph{A.~penninervis}) would be conserved over the
current lectotype (\emph{A.~scorpioides}) of an African taxon
\citep{Orchard05}. Alternate proposals keep the lectotype, and
reclassify some \emph{Acacia} species as `\emph{Racosperma}'
\citep{Luckow05}. `\emph{Acacia}' will be used for the Australian
species in this thesis\footnote{A summary of the events relating to
the renaming of \emph{Acacia} can be found at
\url{http://www.worldwidewattle.com/infogallery/nameissue/chronology.php}}.

\begin{figure} [tb]
    \centering
    \includegraphics[width=12cm]{Acacia}
    \caption[\emph{Acacia longifolia}]{\emph{Acacia longifolia}. Photo \cc Brenda Foran.}
    \label{p-Acacia}
\end{figure}

\emph{Acacia longifolia} (Andrews) Willd. (Sydney golden wattle) is
investigated in this study. It is a serious invasive weed in
Northland, where it was introduced to control sand dune erosion, but
has now spread widely and invaded wetlands \citep{Hicks01}.
\emph{A.~longifolia} is also a significant problem in South Africa
endangering the floristically unique Cape Floral Kingdom
\citep{Dennill99,VanWilgen04}.



\section{Previous research on New Zealand rhizobia}

Work on this project started in early 2002. At this time there were
few reports of rhizobia nodulating native legumes apart from an
Honours dissertation  using a small number of strains
\citep{McCallum96}, and work in the 1960's--70's
\citep{Greenwood69,Greenwood78a,Greenwood78b}. Likewise, there were
no investigations using molecular techniques of rhizobia nodulating
gorse and broom in any country, although historical research lumped
the strains into the inaccurately described `cowpea rhizobia'
\citep{Pieters27,Wilson39a}. It is not until recently that molecular
techniques allowed affordable and accurate assessment of the
phylogeny of bacterial strains.

In early studies, many host-range experiments were done
\citep{Greenwood69,Jarvis77,Crow81}, but interpretation of the data
was difficult, as then the molecular mechanisms behind nodulation
were not known, nor was it known that symbiosis genes were
transmissible. A more comprehensive account of previous
\emph{Rhizobium}--legume research in New Zealand is presented in
Section \ref{s-NZ-rhizobia-history}.

This thesis will build on this previous work, assisted by modern
techniques and knowledge.

\section{Research objectives}

This thesis aims to establish a better understanding of the nature
(taxonomy, diversity, host-range, and distribution) of the
associations between rhizobia and New Zealand's endemic and weed
legume flora.

It is assumed that the native legume genera have co-evolved with
nitrogen-fixing bacterial symbionts for millions of years,
potentially leading to new species. In contrast the origin of
rhizobia nodulating the recently introduced exotic legumes is
unknown. Previous studies overseas \citep[reviewed by][]{Perret00}
have shown that rhizobial strains differ in host-range specificity.
Some (e.g. \emph{Ensifer fredii} NGR234) are promiscuous, while
others appear to be host specific (e.g. \emph{Rhizobium
leguminosarum} bv. \emph{trifolii}). Based on this, there are three
possibilities that could explain exotic legume nodulation: 1)
Introduced legumes are promiscuous and use the same rhizobia as
native legumes. 2) Introduced legumes use specific rhizobia that
were recently introduced---perhaps in conjunction with exotic
legumes. 3) Introduced legumes use specific rhizobia that were
already present in New Zealand (possibly cosmopolitan).

%Either bradyrhizobia exist as ubiquitous free-living autochthons, or
%they are carried with the plants when introduced into pristine
%environments, perhaps with seed.

Specific objectives for this thesis are:

\begin{itemize}

\item To establish the identity and diversity of the rhizobial species associated
with New Zealand's endemic legume species.

\item To establish the identity and presumptive origins of the rhizobial species
associated with the woody legume weeds introduced into New Zealand.

\item To determine the specificity and efficacy of the symbiotic associations of
rhizobial species  with both plant groups, endemic and woody weeds,
by an investigation of their nodulation and nitrogen-fixing
capacity.

\item To investigate possible exchange of transmissible genetic elements
between rhizobial species associated with endemic and introduced
legumes.

\end{itemize}


\section{Research strategy}

To investigate the identity and diversity of rhizobia, a polyphasic
strategy employing both phenotypic and phylogenetic characteristics
was used \citep{Vandamme96}. Phylogenetic analyses were based on the
sequencing of three protein-coding conserved `housekeeping' genes
(\emph{atpD}, \emph{glnII}, \emph{recA}), and one ribosomal RNA gene
(16S). Phenotypic characteristics included metabolic fingerprints
based on substrate utilisation (Biolog), and whole cell fatty acid
methyl ester profiles (FAME).

The symbiosis genes were investigated by sequencing a protein-coding
gene (\emph{nodA}) involved in \emph{Rhizobium}--plant signalling,
which is usually carried on a transmissible genetic element (plasmid
or symbiosis island).

The efficacy of the symbiotic combinations was tested by inoculating
legumes with rhizobial strains in host-range experiments. The
potential to fix nitrogen was determined by acetylene reduction, and
roots were visually examined for nodulation.





%%----------------------FIN-------------------------------------------
