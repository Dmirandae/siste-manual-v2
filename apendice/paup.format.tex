\subsection{PAUP*}
\noindent
Se inicia con la clausula \Cmd{\#nexus}, y luego con el bloque \Cmd{data}. se puede definir si los caractertes son morfol'ogicos, ADN o prote'inas. Los polim'orficos se colocan entre par'entesis redondos. ADN en formato IUPAC.\\
\\
\noindent
\Cmd{\#nexus\\
begin data;\\
dimensions ntax=4 nchar=5;\\
format missing=? gap=- symbols="0 1 2";\\
matrix\\
out          00000\\
alpha       10-20\\
beta        1102(01)\\
gamma    1?111\\
lamda      11111\\
;\\
end;\\
begin assumptions;\\
typeset \*tipoUno=unord:1-3 5, ord:4;\\
end;
\\
begin paup;}\\
\Cmd{
$[$Aqui puede colocar instrucciones espec'ificas de paup, por ejemplo b'usquedas$]$\\
}
\Cmd{hsearch add=random;\\
end;\\
\\
\#nexus\\
begin data;\\
dimensions ntax=4 nchar=5;\\
format missing=? gap=- datatype=dna;\\
matrix\\
out           ACGTC\\
alpha      AT-CG\\
beta  RTAAC\\
gamma    CGAY-\\
lamda      TCNCC\\
;\\
end;
}