\subsection{NONA}
\noindent
\Pname{NONA}, \Pname{Hennig86}, \Pname{WinClada} y \Pname{TNT} usan este formato; con \cmd{*} indican que hay m\'as \'arboles y con \cmd{;} que es el \'ultimo \'arbol. N\'otese el espacio para separar los terminales. El primer taxon es 0. Al igual que en las matrices, \cmd{p-} o \cmd{p/} indican el final de lectura del archivo. Winclada puede incluir al incio la lista de nombres, pero no es compatible con otros programas.\\
\\
\noindent
\cmd{
tread $'$tres arboles$'$\\
(0 (1 (2 (3 4 ))))*\\
(0 (1 (2 3 4 )))*\\
(0 ((1 2 )(3 4 )));\\
p/;}\\
\\
\cmd{
tread $'$solo un arbol$'$\\
(0 (1 (2 (3 4 ))));\\
p/;}
\subsection{PAUP*}
En \Pname{PAUP*} el \'arbol est\'a embebido en el archivo de la matriz o en un formato aparte. Los grupos son separados por comas y  el primer taxon es 1.\\
\\
\noindent
\cmd{
\#nexus\\
\\
begin trees;\\
  translate\\
    1 lamda,\\
    2 alpha,\\
    3 beta,\\
    4 gamma,\\
    5 out\\
  ;\\
  tree *primero=(5,(1,(2,(3,4))));\\
  tree politomico=(5,(1,(2,3,4)));\\
  tree tercero=(5,((1,2),(3,4)));\\
end;}\\
\\
El orden no altera los \'arboles en los programas. As\'i:\\
\cmd{(0,(1,(2,(3,4))))}\\
es igual a\\
\cmd{((1,((3,4),2)),0)}
