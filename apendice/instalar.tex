\subsection{WinClada}
\noindent


Plataforma: Windows 9x o superior [32 o 64 BITS acorde con el computador en uso].

En est\'a secci\'on tenemos los com\'andos b\'asicos para instalar varios de los programas que se usan durante el curso:


\Pname{Winclada}\url{http://www.diversityoflife.org/winclada/}

 Disponibilidad: Gratuito. 

 
 

\Pname{Nona[Piwe]}\url{http://www.lillo.org.ar/phylogeny/Nona-PeeWee/}

 Disponibilidad: Gratuito. 

 

\Pname{TNT}\url{http://www.lillo.org.ar/phylogeny/tnt/}

Disponibilidad: Gratuito. 

 
\Pname{IQTree}\url{http://www.iqtree.org/}

Disponibilidad: Gratuito. 

 
\Pname{MrBayes}\url{https://nbisweden.github.io/MrBayes/download.html}

Disponibilidad: Gratuito. 
 

Tambi\'en puede usar navegador preferido y  buscar el programa en caso de que los enlaces no estén disponibles.


Para todos los casos, los pasos sugeridos son:

 

\begin{enumerate}
\item Crear un directorio o carpeta para  guardar los archivos 
\item Ir al sitio red de cada programa
\item Salvarlo
\item en caso de que sea necesario unzip en el mismo sitio
\end{enumerate}

estos programas no tien un ''instalador'' formal, por lo que es recomendable guardarlo en us sitios conocido y ejecutarlo desde tal directorio, donde, preferiblemente, tambien están los datos.



Mucohs de estos programas se manejan por consola [o l\'inea de comandos] por lo que para usarlos se debe seguir la secuencia

\begin{itemize}
\item cmd.exe [o equivalente]
\item ir hasta el sitio conocido [cd D: etc]
\item ''llamar'' el programa por ejemplo: ./iqtree.exe
\item y en los casos posibles revisar la ayuda  [./iqtree.exe -h]
\end{itemize}
