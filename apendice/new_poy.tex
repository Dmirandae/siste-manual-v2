\subsection{POY}
\noindent
Autor: Var\'on et al., 2007.

Plataforma: Linux, MacOS X y Windows.

Disponibilidad: gratuito, \url{research.amnh.org/scicomp/projects/POY.php}


Es el \'unico programa disponible que realiza un an\'alisis simult\'aneo sin utilizar alineamiento previo de las secuencias moleculares. Tambi\'en puede usarse para realizar alineamientos, o para hacer b\'usquedas convencionales de datos previamente alineados o morfol\'ogicos. La versi\'on actual posee \'unicamente una implementaci\'on de parsimonia, pero se planean versiones que tambi\'en realicen b\'usquedas bajo m\'axima verosimilitud.

El programa posee cuatro ventanas: una de salida, donde se imprimen salidas pedidas por el usuario (por ejemplo, un \'arbol, o la ayuda). Una ventana de comandos, donde se escriben las \'ordenes al programa, una donde muestra que clase de tarea se esta realizando y en la \'ultima donde se muestra el progreso de las b\'usquedas.

Las matrices y arboles son abiertos usando \cmd{read($"$archivo$"$)}, donde archivo es el archivo a leer. Puede leer m\'ultiples conjuntos de datos, bien sea abriendo uno por uno, o varios en la misma instrucci\'on como \cmd{read($"$arch1$"$,$"$arch2$"$)}. Los datos se van agregando. Para limpiar la memoria se utiliza \cmd{wipe()}. Se buscan \'arboles con \cmd{build(x)}, que realiza x arboles de Wagner. Para mejorar dichos \'arboles hay que usar \cmd{swap()}. Note que las instrucciones est\'an separadas. Es posible hacer b\'usquedas m\'as sofisticadas usando nuevas tecnolog\'ias, como ratchet implementado en \cmd{perturb(iterations: x, ratchet())}, donde se har\'ian x iteraciones de ratchet, con \cmd{swap(drift)} o \cmd{swap(annealing)} se hace deriva de \'arboles y cristalizaci\'on simulada, y \cmd{fuse()} hace fusi\'on de \'arboles.

El comando para manejar costos es \cmd{transform(tcm())}. Se puede modificar los costos de transformacion para datos $"$est\'aticos$"$ como morfolog\'ia con 

\Cmd{transform(static, weight:2)} 
que dar\'ia un peso de 2 a cada transformaci\'on de datos morfol\'ogicos. Con la instrucci\'on:

\Cmd{transform(tcm(1,2))} 

se da un peso de 1 a las sustituciones y de 2 a los gaps y caracteres est\'aticos, ese es el valor que viene por omisi\'on. Con  la instrucci\'on:

\Cmd{transform(fixedstates)} 

se implementa el m\'etodo de estados fijos de Wheeler.


Es posible calcular soportes con 
\cmd{calculate\_support()}, 
que incluye \textit{bootstrap}, \textit{jackknife} y soporte de Bremer (por omisi\'on). Las salidas se realizan con 
\cmd{report($"$archivo$"$)}, 
 pueden ser cladogramas en formato de par\'entesis, dibujados en ascii, o en \textit{postscript}. Tambi\'en es \'util para conocer estad\'isticas de los \'arboles, las optimaciones de los nodos y la integridad de los datos.
 
El programa funciona desde computadoras de escritorio hasta \textit{clusters}. Puede ejecutarse directamente en la l\'inea de comando o en procesos de lotes.
