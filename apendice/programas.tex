\chapter{Programas de c\'omputo}
\label{ch:programas}

Hoy en d\'ia los an\'alisis filogen\'eticos usan decenas, cientos a miles de terminales y de caracteres, por lo que se requiere gran  cantidad de c\'alculos. Aunque es posible hacerlos a mano, esto se 
hace imposible para m\'as de 10 taxa y una veintena de caracteres. 
Esta secci\'on busca familiarizar al estudiante con los programas 
m\'as comunes, aparte de los usados durante las distintas pr\'acticas. 
Es casi un truismo que la selecci\'on de programas depende de la(s) 
plataforma(s) utilizada(s), el problema a resolver, y en menor grado 
de los fondos disponibles, gratis o $"$por unos pocos d\'olares$"$ se 
puede tener un repertorio apropiado de programas, la condici\'on 
central a tener en cuenta es la \textbf{velocidad} y la \textbf{eficiencia}. 
La \textbf{facilidad de manejo} en la gran mayor\'ia de casos es secundaria; 
aunque la curva de aprendizaje puede ser lenta y tortuosa, las 
habilidades ganadas hacen que los an\'alisis en el largo plazo sean 
m\'as f\'aciles de ejecutar; a eso se suma la posibilidad de manejar 
$"$lenguajes$"$ por casi todos los programas. Es posible que al final 
del proceso todo sea asunto de un par de instrucciones, una vez 
que ha perfeccionado sus habilidades y que posee archivos de 
instrucciones preajustadas a sus gustos y necesidades. 

En la medida de lo posible, familiar\'icese con la l\'inea de comandos, casi todos los programas interactuan de esa manera y lo aprendido le permitir\'a analizar los datos en un menor tiempo ya que podr\'a hacer an\'alisis a medida, m\'as alla de lo que permita la interfaz gr\'afica y podr\'a hacer los c\'alculos en paralelo, algo que no es f\'acil o posible desde la interfaz gr\'afica.

\newpage

\section{Editores y manejadores de b\'usqueda}
\index{Programas!Edici\'on de matrices}
\input{./apendice/Winclada.tex}
\input{./apendice/McClade.tex}
\section{B\'usqueda de \'arboles}
\index{Programas!b\'usquedas}
\subsection{TNT}
\noindent
Autor: \cite{tnt}, plataforma: Windows, MacOS, Unix.

Disponibilidad: Gratis, es posible descargarlo desde \url{http://www.lillo.org.ar/phylogeny/tnt/}.

Para parsimonia, es el programa m\'as r\'apido que se ha desarrollado; incluye varios tipos de b\'usquedas especializadas, como la deriva y la fusi\'on de \'arboles.
Al igual que \Pname{NONA}, \cmd{proc} le permite abrir la matriz de datos o archivos de instrucciones. Para las b\'usquedas puede configurar los diferentes m\'etodos usando \cmd{ratchet}, \cmd{drift} y \cmd{mult}; al usar \cmd{?} puede obtener los par\'ametros, y con \cmd{=} puede modificarlos.
\cmd{Mult} puede usarse como la $"$central de b\'usqueda$"$, definiendo un n\'umero de r\'eplicas para una b\'usqueda de Wagner y las posteriores mejoras con ratchet y drift (seg\'un como est\'en configurados). Para correr simplemente escriba \cmd{mult: replic X;} para ejecutar el n\'umero de r\'eplicas que desee (X), con \cmd{tsave* nombre} se abre el archivo $"$nombre$"$ para guardar \'arboles. Gu\'ardelos con \cmd{save} y al final no olvide cerrar el archivo: \cmd{tsave/;}.

El programa cuenta con ayuda en l\'inea que puede ser consultada usando \cmd{help} o escribiendo \cmd{help comando} para un comando espec\'ifico. Este es \'unico programa que tiene directamente implementado el remuestreo sim\'etrico, adem\'as tiene implementado el soporte de Bremer, el soporte relativo de Bremer, soporte de Bremer dentro de los l\'imites del Bremer absoluto y puede calcular la cantidad de grupos soportados-contradichos (FC), con \cmd{resample} usted puede configurar la permutaci\'on. Con \cmd{subop} (tambi\'en en \Pname{NONA}) usted puede indicar el costo de los sub\'optimos, en diferencia de pasos con respecto al \'arbol \'optimo. \cmd{Bsupport} hace el c\'alculo del \'indice de Bremer, con un \cmd{*} calcula el valor relativo, o puede usar \cmd{Bsupport ]} para calcular el soporte relativo dentro de los l\'imites del absoluto. Los resultados se pueden almacenar usando algunas herramientas del lenguaje de macros\footnote{\url{www.lillo.org.ar/phylogeny/tnt/scripts/General_Documentation.pdf}}.


\subparagraph*{Versi\'on de men\'u}
En  Windows es bastante similar a \Pname{winClada}, pero esta mucho mejor organizada. Algunas funciones son:

\begin{itemize}
	\item \Gui{Settings}: Adem\'as de las diferentes opciones de macros, manejo de memoria, tambi\'en contiene los par\'ametros usados en el colapsado de ramas, consensos, y pesado impl\'icito.
	\item \Gui{Analyze}: Contiene los diferentes tipos de b\'usquedas y sus par\'ametros, el manejo de \'arboles sub\'optimos, y los remuestreos.
	\item \Gui{Optimize}: Permite ver y dibujar sinapomorf\'ias, mapear caracteres, y revisar estad\'isticos de caracteres y \'arboles (por ejemplo long o peso).
	\item\Gui{Trees}: All\'i se encuentran las entradas para el dibujo de \'arboles, el c\'alculo soporte de Bremer, realizar consensos y super-\'arboles, as\'i como \'arboles al azar, y el manejador de etiquetas de los nodos (\textit{tags}).
	\item \Gui{Data}: Aparte de la configuraci\'on de caracteres y terminales, posee un editor b\'asico de datos, que aunque muy simple, es extremadamente f\'acil de manejar, y m\'as directo que los editores basados en mostrar la matriz (Como \Pname{winClada}, o \Pname{Mesquite}).
\end{itemize}

\subsection{NONA}
\noindent
Autor: Goloboff, 1998.

Plataforma: Linux, Mac o Windows (9x o superior).

Disponiblilidad: Gratuito en conjunto con otros programas en \url{www.lillo.org.ar/phylogeny/Nona-PeeWee/}.


Este programa es una buena opci\'on que existe para b\'usquedas bajo parsimonia, dado que es gratuito, tiene lenguaje de macros y es veloz, pero dados los costos, velocidad y operatibidad de \Pname{TNT}, este \'ultimo es la mejor opci\'on.


B\'asicamente los comandos de \Pname{NONA} son los mismos de \Pname{TNT}. Una diferencia importante (aparte de la velocidad y algoritmos implementados) es que en \Pname{NONA} los comandos son ejecutados (en vez de poder configurar sin ejecutar). A diferencia de \Pname{PAUP*}, no puede desactivar terminales.

Al igual que en \Pname{TNT}, el comando simple de b\'usqueda es \cmd{mult}. Para permutar ramas se utiliza \cmd{max}. En ambos casos debe colocarse un asterisco \cmd{mult*; max*} para que utilice TBR como permutaci\'on, en caso contrario usar\'a SPR.


Para salvar el \'arbol, el comando es \cmd{sv}. La primera vez que se llama, solo abre el archivo (Debe darse como par\'ametro o el programa solicita un nombre). Con \cmd{sv*} salva todos los \'arboles en memoria. Usando \cmd{ksv} se guardan los \'arboles colapsados, de lo contrario siempre se guardan dicot\'omicos. Para salvar el consenso es necesario usar \cmd{inters}.

El \'unico m\'etodo de soporte expl\'icitamente implementado es el soporte de Bremer \cmd{bsupport}: con asterisco \cmd{bs*} muestra los soportes relativos, \'o de lo contrario muestra el valor absoluto. Pero el programa viene acompa\~nado de varios programas y macros que permiten calcular \textit{jackknife} y \textit{bootstrap}, as\'i como hacer medidas basadas en frecuencias relativas (FC).
\input{./apendice/new_POY.tex}
\input{./apendice/paup.tex}
\subsection{MrBayes}
\noindent
Autor: Ronquist et al., 2005.\\
Plataforma: Windows, MacOS, Unix.\\
Disponibilidad: gratuito, \url{http://mrbayes.net}

Es el programa m\'as com'unmente usado para an'alisis bayesiano. Es gratuito y su interfaz es muy similar a la de PAUP*. Los principales comandos del programa son descritos en el cap\'itulo de an'alisis bayesiano. El tutorial incluido es muy completo y viene dentro del programa. Una de las caracter\'isticas m\'as interesantes es la implementaci\'on de los $"$modelos$"$ usados para caracteres morfol\'ogicos (Lewis, 2001), as\'i como el modelo  $'$parsimonia$'$ tal y como fue descrito por \cite{tuffley1997} que se activa usando 

\Cmd {lset parsmodel=yes;}






%\section{Miscel\'aneos}
%\input{./apendice/component2.tex}
%\input{./apendice/Modeltest.tex}
%\input{}


