\chapter{Programas de c\'omputo}
\label{ch:programas}

Hoy en d\'ia los an\'alisis filogen\'eticos usan decenas, cientos a miles de terminales y de caracteres, por lo que se requiere gran  cantidad de c\'alculos. Aunque es posible hacerlos a mano, esto se 
hace imposible para m\'as de 10 taxa y una veintena de caracteres. 
Esta secci\'on busca familiarizar al estudiante con los programas 
m\'as comunes, aparte de los usados durante las distintas pr\'acticas. 
Es casi un truismo que la selecci\'on de programas depende de la(s) 
plataforma(s) utilizada(s), el problema a resolver, y en menor grado 
de los fondos disponibles, gratis o $"$por unos pocos d\'olares$"$ se 
puede tener un repertorio apropiado de programas, la condici\'on 
central a tener en cuenta es la \textbf{velocidad} y la \textbf{eficiencia}. 
La \textbf{facilidad de manejo} en la gran mayor\'ia de casos es secundaria; 
aunque la curva de aprendizaje puede ser lenta y tortuosa, las 
habilidades ganadas hacen que los an\'alisis en el largo plazo sean 
m\'as f\'aciles de ejecutar; a eso se suma la posibilidad de manejar 
$"$lenguajes$"$ por casi todos los programas. Es posible que al final 
del proceso todo sea asunto de un par de instrucciones, una vez 
que ha perfeccionado sus habilidades y que posee archivos de 
instrucciones preajustadas a sus gustos y necesidades. 

En la medida de lo posible, familiar\'icese con la l\'inea de comandos, casi todos los programas interactuan de esa manera y lo aprendido le permitir\'a analizar los datos en un menor tiempo ya que podr\'a hacer an\'alisis a medida, m\'as alla de lo que permita la interfaz gr\'afica y podr\'a hacer los c\'alculos en paralelo, algo que no es f\'acil o posible desde la interfaz gr\'afica.

\newpage

\section{Editores y manejadores de b\'usqueda}
\index{Programas!Edici\'on de matrices}
\subsection{WinClada}
\noindent
Autor: Nixon, 2002.

Plataforma: Windows 9x o superior [Funciona MUY bien en \Pname{wine} dentro de Linux].

Disponibilidad: Shareware (30 d\'ias de prueba), tiene un costo de US 50.00, 
\url{http://www.cladistics.com}.

El programa require de \Pname{NONA} para realizar las b\'usquedas (\Pname{PIWE} o \Pname{Hennig86} son deseables, pero no obigatorios).

Aunque es un programa obsoleto, es uno de los programas m\'as \'utiles, tanto para principiantes como para iniciados. El programa tiene dos entornos para manejar matrices y \'arboles. El manejo e impresi\'on de \'arboles permite que se tengan im\'agenes listas para publicaci\'on con pocos pasos y sin necesidad de retoque posterior. Tiene m\'ultiples $"$conflictos$"$ con el manejo de im\'agenes que pueden ser inc\'omodos; adem\'as, el usuario debe tener presente que el programa numera por defecto (taxa, caracteres y \'arboles) desde cero (la manera l\'ogica para los programadores), o desde uno, por lo que se debe tener cuidado al escribir el texto y referenciar la gr\'afica.

\subparagraph*{WinDada}
\Gui{Matrix}: Le permite cambiar algunos aspectos de la matriz (crearlas, cambiar tama\~no, salvarlas, fusionarlas).

\Gui{Edit}: Lo m\'as importante es que permite que los datos sean o no modificables. Adem\'as, permite adicionar polim'orfismos.

\Gui{Terms/Chars}: Estos men\'u le permiten modificar cosas espec\'ificas de terminales y caracteres, tales como nombre y orden, seleccionarlos, borrarlos, adicionar.

\Gui{View}: Opciones de presentaci\'on. Adem\'as, puede ver estad\'isticos de los caracteres, aditivos, activos, pasos m\'inimos y m\'aximos; y le permite intercambiar modo num\'erico o IUPAC (para DNA).

\Gui{Output}: Le permite exportar en formatos diferentes, o informaci\'on variada sobre los caracteres.

\Gui{Key}: Si tiene los caracteres nominados, funciona como clave interactiva.

\Gui{Analize}: Tiene diferentes entradas para b\'usquedas y c\'alculo de soporte. En \Gui{heuristics}, es posible configurar la cantidad m\'axima de \'arboles a retener, la cantidad de \'arboles por cada r\'eplica (\'arbol de Wagner+TBR) (i.e. estrategias NONA/PAUP) y otros detalles de la b\'usqueda, como especificar una semilla para los n\'umeros aleatorios (0 es el tiempo interno de la computadora). En \Gui{ratchet}: se puede modificar la cantidad de iteraciones de ratchet, el n\'umero de \'arboles a retener por iteraci\'on y el n\'umero de b\'usquedas de ratchet, bien sean secuenciales o simult\'aneas. Mientras que con \Gui{bootstrap/jackknife/CR with NONA} permite manipular las diferentes opciones para hacer soporte con remuestreo. Recuerde borrar todos los \'arboles antes de ejecutar la b\'usqueda de remuestreo.

\subparagraph*{CPanel}
Para llenar la matriz.
\Gui{Mode}: Cambia el modo de acceso de los caracteres.
\subparagraph*{Winclados}
Para manejar el \'arbol. En general, la mayor parte de estas acciones son accesibles desde la barra de herramientas con el rat\'on.

Las teclas F2-F12 tienen (con y sin SHIFT) diversas funcionalidades de visualizaci\'on, como engrosar/adelgazar el ancho de las ramas, expandir/contraer el costo, moverse entre \'arboles.

\Gui{Trees}: opciones para visualizar y manipular el \'arbol sin modificar su topolog\'ia (forma, consensos, ir a un \'arbol determinado).

\Gui{Nodes}: Para ver frecuencias de nodos, desafortunadamente su funcionalidad es muy inestable, y a veces produce resultados inesperados.

\Gui{HashMarks}: Para ver las transformaciones en los nodos.

\Gui{Edit/Mouse modes:} Modifica las acciones posibles del rat\'on para modificar el \'arbol, como mover nodos, colapsarlos, cambiar la ra\'iz.

\Gui{Diagnoser}: Para mapear caracteres.

\subsection{MacClade \& Mesquite}
\noindent
Autores: Maddison \& Maddison, 2005.\\
Plataforma (\Pname{McClade}): Macintosh (MacOS X, PPC y Classic 68k. Esta 'ultima funciona bien con emuladores como BasiliskII)\\
Plataforma (\Pname{Mesquite}): Cualquiera, requiera Java virtual machine.\\
Disponibilidad (\Pname{McClade}): Descontinuado pero puede acceder desde el sitio de los autores para el programa que funciona en OS X (hasta 10.6):\\
http://macclade.org/index.html.
\\
Disponibilidad (\Pname{Mesquite}): Gratuito.\\ http://mesquiteproject.org.
\\
\paragraph*{}
\Pname{MacClade} es un programa cuyas cualidades gr'aficas son impresionantes, su interfase es agradable, sencilla y no transmite miedo a los novatos. Dadas sus magnificas propiedades gr'aficas, es \textbf{muy} recomendable para la edici'on de 'arboles y matrices, pero principalmente para la optimizaci'on de caracteres, la cual permite distintos tipos de cambio de los mismos (v.g., Dollo, ACTRAN, DELTRAN) y adicionalmente tiene una caracter'istica 'unica de mapeo, el ''equivocal ciclying'' [Maddison \& Maddison, 1992].\\
En \Pname{MacClade} la edici'on de los datos y el manejo de los 'arboles se llevan a cabo en dos interfaces separadas para 'arboles y matrices, las cuales son accesibles en la ventana \Gui{Windows}. Dado que Macclade esta dise\~nado para ser igualmente funcional con datos morfol'ogicos y moleculares,  existen m'ultiples herramientas de edici'on.\\
\subparagraph*{Data Editor}
En esta interfaz se construye y edita la matriz. Sus ventanas son:\\
\Gui{Edit}: se pueden duplicar caracteres o taxa, editar bloques de comandos que corran directamente en PAUP. \\
\Gui{Utilities}: se pueden buscar secuencias particulares dentro de la secuencia total, reemplazar caracteres por otros, reemplazar
datos ausentes por \textit{gaps} y viceversa, importar alineamientos desde el \textit{genbank} y finalmente se puede lograr que la matriz hable por si sola !`s'i, que hable! Un lector autom'atico lee en forma descendente los taxa de su matriz facilit'andote un poco el trabajo (puede usar de hecho ingl'es brit'anico, si esto lo hace m'as feliz).\\
\Gui{Characters}: permite adicionar caracteres, incluir los estados, ver una lista de los caracteres que se han incluido, determinar el formato de los caracteres que est'an en la matriz (si son prote'inas, nucle'otidos, o est'andar, que es el definido para simbolog'ia de n'umeros en las casillas); tambi'en permite determinar el tipo de cambio que se permitir'a para los caracteres (cuando se est'a haciendo una optimizaci'on), el peso de los caracteres.\\
\Gui{Taxa}: permite hacer cosas semejantes a \Gui{characters}, pero para el manejo de los taxa, incluir taxa nuevos, crear listas de taxa o reordenarlos. \\
\Gui{Display}: finalmente esta ventana maneja toda la configuraci'on gr'afica de la matriz, el tipo de letra, su tama\~no, el color de los caracteres, el ancho de columna, etc.\\
La \'ultima ventana en el men'u es \Gui{Windows}, el cual permite moverse entre las interfaces de los datos y el 'arbol y llamar a una caja de herramientas que por omisi'on siempre est'a presente en la parte inferior izquierda de la ventana. Esta caja ofrece herramientas como llenar estados, expandir columnas, cortar, entre otras. En esta ventana \Gui{notes about trees} y \Gui{note file} permiten incluir comentarios acerca de los 'arboles y sobre la matriz.
\subparagraph*{Tree Window}
Esta interfaz maneja y modifica los 'arboles y posee las ventanas b'asicas del \Gui{data editor}, no obstante, incluye otras nuevas, espec'ificas para el mapeo de caracteres y manejo de los 'arboles. Solo se describen aquellas nuevas ventanas: \\
\Gui{Trees}: Esta ventana permite cambiar, abrir archivos externos para incluir 'arboles, crear una lista de los 'arboles que esta incluidos a la matriz, guardar los 'arboles, exportarlos en formato de hennig86 o Phylip (ver anexo de formatos para m'as informaci'on) y manejar las politomias como politomias blandas o duras (Coddington \& Scharff, 1994).\\
\Gui{$\Sigma$}: esta ventana incluye el manejo de todos los estad'isticos  del 'arbol, como la longitud, el 'indice de consistencia, 'indice de retenci'on, 'indice de consistencia escalonado, el n'umero de cambios en el 'arbol y finalmente, puede generarse el archivo de comandos para correr el indice \Gui{decay}, 'o soporte de Bremer en \Pname{PAUP*} (vea el cap'itulo sobre programas).\\
% aqui debe ponerse una etiqueta dirigida a programas paup
\Gui{Trace}: aqu'i se maneja todo lo relacionado con el mapeo de los caracteres; puede mapear todos los caracteres a la vez, o escoger un determinado car'acter para ser mapeado. Tambi'en se escoge el tipo de cambio de los caracteres (\Gui{resolving options}) el cual puede ser ACTRAN o DELTRAN, y escoger el tipo de mapeo para los caracteres continuos, \Pname{MacClade} y \Pname{Mesquite} son los 'unicos programas que permiten mapear tal tipo de caracteres.\\
\Gui{Chart}: En esta ventana se pueden generar cuadros de estad'isticas de los caracteres vs pasos y 'arboles (characters states/etc.) o de los cambios de caracteres, la ocurrencia de los estados  a trav'es de todos los caracteres. Finalmente puede comparar dos 'arbolesresolviendo sus politomias o tal y como son.\\
\Gui{Display}: finalmente en esta ventana se manejan los aspectos gr'aficos de la resentaci'on de los 'arboles tales como el tipo de letra, el tama\~no y el estilo de la letra, el estilo y forma del 'arbol, numeraci'on de las ramas, el tama\~no del 'arbol, la configuraci'on de los colores  del mapeo y la simbolog'ia del los taxa (como n'umeros o nombres).\\
\\
Aunque con menos capacidades que \Pname{McClade}, \Pname{Mesquite} es una herramienta poderosa, teniendo en cuenta que es gratuito. Su principal falla es que no permite ver las transformaciones en los nodos de todos los caracteres simult'aneamente. Su interfaz es muy similar a la de \Pname{McClade} y la distribuci'on de ventanas y comandos es m'as o menos equivalente.\\

\section{B\'usqueda de \'arboles}
\index{Programas!b\'usquedas}
\subsection{TNT}
\noindent
Autor: Goloboff et al., 2007.\\
Plataforma: MS-DOS, Windows, MacOS, Unix.\\
Disponibilidad: es posible descargar un demo funcional, para 10 corridas. 
El programa tiene un costo de US80.00.\\
\href{}http://www.zmuc.dk/public/phylogeny/TNT/.
\\
Para parsimonia, es el programa m'as r'apido que se ha desarrollado; incluye varios tipos de b'usquedas especializadas, como la deriva y la fusi'on de 'arboles.\\
\\
Al igual que \Pname{NONA}, \Cmd{proc} le permite abrir la matriz de datos o archivos de instrucciones. Para las b'usquedas puede configurar los diferentes m'etodos usando \Cmd{ratchet}, \Cmd{drift} y \Cmd{mult}; al usar \Cmd{?} puede ver c'uales son los parametros, y con \Cmd{=} puede modificarlos.\\ \Cmd{Mult} puede usarse como la ''central de b'usqueda'', definiendo un n'umero de r'eplicas para una b'usqueda de Wagner y las posteriores mejoras con ratchet y drift (seg'un como est'en configurados). Para correr simplemente escriba \Cmd{mult: replic X;} para ejecutar el n'umero de replicas que desee (X).\\
Con \Cmd{tsave* nombre} se abre el archivo ''nombre'' para guardar 'arboles. Gu'ardelos con \Cmd{save} y al final no olvide cerrar el archivo: \Cmd{tsave/;}\\
El programa cuenta con ayuda en l'inea que puede ser consultada usando \Cmd{help} o escribiendo \Cmd{help comando} para un comando espec'ifico.\\
Este es 'unico programa que tiene directamente implementado el remuestreo sim'etrico. Adem'as tiene implementado el soporte de Bremer, el soporte relativo de Bremer, el soporte de Brtemer dentro de los l'imites del Bremer absoluto y puede calcular la cantidad de grupos soportados-contradichos (FC). Con \Cmd{resample} usted puede configurar como quiere la permutaci'on. Con \Cmd{subop} (tambi'en en \Pname{NONA}) usted puede indicar la longitud de los sub'optimos, en diferencia de pasos con respecto al 'arbol 'optimo. \Cmd{Bsupport} hace el c'alculo del 'indice de Bremer, con un \Cmd{*} calcula el valor relativo, o puede usar \Cmd{Bsupport ]} para calcular el soporte relativo dentro de los l'imites del absoluto. Infortunadamente los resultados no se almacenan en ninguna parte, por lo que debe generar una salida (en formato de texto).\\
\subparagraph*{Versi'on de men'u}
Solo para Windows es bastante similar a \Pname{winClada}, pero esta mucho mejor organizada. Algunas funciones son:\\
\Gui{Settings}: Adem'as de las diferentes opciones de macros, manejo de memoria, tambi'en contiene los par'ametros usados en el colapsado de ramas, consensos, y pesado impl'icito.\\
\Gui{Analyze}: Contiene los diferentes tipos de b'usquedas y sus par'ametros, el manejo de 'arboles suboptimos, y los remuestreos.\\
\Gui{Optimize}: Permite ver y dibujar sinapomorf\'ias, mapear caracteres, y revisar estad\'isticos de caracteres y 'arboles (por ejemplo long o peso).\\
\Gui{Trees}: All'i se encuentran las entradas para el dibujo de arboles, el calculo soporte de Bremer, realizar consensos y super-'arboles, as'i como 'arboles al azar, y el manejador de etiquetas de los nodos (\textit{tags}).\\
\Gui{Data}: Aparte de la configuraci\'on�de caracteres y terminales, posee un editor b'asico de datos, que aunque muy simple, es extremadamente f'acil de manejar, y mas directo que los editores basados en mostrar la matriz (Como \Pname{winClada}, o \Pname{Mesquite}).\\
Usted puede encontrar un manual para el lenguaje de macros en la direcci\'on:\\ 
\href{}http://www.zmuc.dk/public/phylogeny/TNT/scripts/
\subsection{NONA}
\noindent
Autor: Goloboff, 1998.

Plataforma: Linux, Mac o Windows (9x o superior).

Disponiblilidad: Gratuito en conjunto con otros programas en \url{www.lillo.org.ar/phylogeny/Nona-PeeWee/}.


Este programa es una buena opci\'on que existe para b\'usquedas bajo parsimonia, dado que es gratuito, tiene lenguaje de macros y es veloz, pero dados los costos, velocidad y operatibidad de \Pname{TNT}, este \'ultimo es la mejor opci\'on.


B\'asicamente los comandos de \Pname{NONA} son los mismos de \Pname{TNT}. Una diferencia importante (aparte de la velocidad y algoritmos implementados) es que en \Pname{NONA} los comandos son ejecutados (en vez de poder configurar sin ejecutar). A diferencia de \Pname{PAUP*}, no puede desactivar terminales.

Al igual que en \Pname{TNT}, el comando simple de b\'usqueda es \cmd{mult}. Para permutar ramas se utiliza \cmd{max}. En ambos casos debe colocarse un asterisco \cmd{mult*; max*} para que utilice TBR como permutaci\'on, en caso contrario usar\'a SPR.


Para salvar el \'arbol, el comando es \cmd{sv}. La primera vez que se llama, solo abre el archivo (Debe darse como par\'ametro o el programa solicita un nombre). Con \cmd{sv*} salva todos los \'arboles en memoria. Usando \cmd{ksv} se guardan los \'arboles colapsados, de lo contrario siempre se guardan dicot\'omicos. Para salvar el consenso es necesario usar \cmd{inters}.

El \'unico m\'etodo de soporte expl\'icitamente implementado es el soporte de Bremer \cmd{bsupport}: con asterisco \cmd{bs*} muestra los soportes relativos, \'o de lo contrario muestra el valor absoluto. Pero el programa viene acompa\~nado de varios programas y macros que permiten calcular \textit{jackknife} y \textit{bootstrap}, as\'i como hacer medidas basadas en frecuencias relativas (FC).
\subsection{Poy}
\noindent
Autor: Var'on et al., 2007.\\
Plataforma: Linux, MacOS y Windows.\\
Disponibilidad: gratuito.\\
\href{}http://research.amnh.org/scicomp/projects/poy.php
\\
Es el 'unico programa disponible que realiza un an'alisis simult\'aneo sin utilizar alineamiento previo de las secuencias moleculares. Tambi'en puede usarse para realizar alineamientos, o para hacer b'usquedas convencionales de datos previamente alineados o morfol'ogicos. La versi'on actual posee 'unicamente una implementaci'on de parsimonia, pero se planean versiones que tambi'en realicen b'usquedas bajo m'axima verosimilitud.\\
\\
El programa posee cuatro ventanas: una de salida, donde se imprimen salidas pedidas por el usuario (por ejemplo, un 'arbol, o la ayuda). Una ventana de comandos, donde se escriben las ordenes al programa, una donde muestra que clase de tarea se esta realizando y en la \'ultima donde se muestra el progreso de las b'usquedas.\\
Las matrices y arboles son abiertos usando \Cmd{read(''archivo'')}, donde archivo es el archivo a leer. Puede leer multiples conjuntos de datos, bien sea abriendo uno por uno, o varios en la misma instrucci'on como \Cmd{read(''arch1'',''arch2'')}. Los datos se van agregando. Para limpiar la memoria se utiliza \Cmd{wipe()}.\\
Se buscan 'arboles con \Cmd{build(x)}, que realiza x arboles de Wagner. Para mejorar dichos \'arboles hay que usar \Cmd{swap()}. Notese que las instrucciones est'an separadas. Es posible hacer b'usquedas m'as sofisticadas usando nuevas tecnolog'ias, como ratchet implementado en \Cmd{perturb(iterations: x, ratchet())}, donde se har'ian x iteraciones de rathcet, con \Cmd{swap(drift)} o \Cmd{swap(annealing)} se hace deriva de 'arboles y cristalizaci'on simulada, y \Cmd{fuse()} hace fusi'on de 'arboles.\\
El comando para manejar costos es \Cmd{transform(tcm())}. Se puede modificar los costos de transformacion para datos ''est'aticos'' como morfolog'ia con 

\Cmd{transform(static, weight:2)} 

que dar'ia un peso de 2 a cada transformaci'on de datos morfol'ogicos. Con 

\Cmd{transform(tcm(1,2))} 

se da un peso de 1 a las sustituciones y de 2 a los gaps y caracteres est'aticos, ese es el valor que viene por omisi'on. Con 

\Cmd{transform(fixedstates)} 

se implementa el m'etodo de estados fijos de Wheeler.


Es posible calcular soportes con 

\Cmd{calculate\_support()} 

que incluye \textit{bootstrap}, \textit{jackknife} y soporte de Bremer (por omisi'on). Las salidas se realizan con 

\Cmd{report(''archivo'')} 

 pueden ser cladogramas en formato parentical, dibujados en ascii, o en \textit{postscript}. Tambi'en es 'util para conocer estad'isticas de los 'arboles, las optimaciones de los nodos y la integridad de los datos.
 
 
El programa funciona desde computadoras de escritorio hasta \textit{clusters}. Puede ejecutarse directamente en la l\'inea de comando o en procesos de lotes.

\subsection{PAUP*}
\noindent
Autor: Swofford, 2002.\\
Plataforma: MS-DOS, Windows, MacOS, Unix.\\
Disponibilidad, el programa tiene un costo entre US 80.00 a US150.00 dependiendo del sistema operativo; es distribuido por Sinauer.\\
http://www.sinauer.com/detail.php?id=8060.
\\
\paragraph*{}
\Pname{PAUP*} es uno de los programas m'as usados en cuanto a b'usquedas mediante ML, o si el sistema operativo es Mac. Excepto para esta 'ultimo plataforma, la interfaz dominante es por l'inea de comandos; la interfaz para Windows en modo gr'afico ni siquera es un p'alido reflejo de la de Mac; como \Pname{NONA} y \Pname{TNT} tiene la opci'on de manejar un gran n'umero de parametros que permiten hacer una b'usqueda \textbf{sobre pedido}, adicionalmente realiza todos los consensos directamente sin necesitar un segundo programa. Su gran desventaja es su velocidad y la falta de un lenguaje de macros.\\
\Cmd{set}:esta funci'on define la configuraci'on b'asica de algunos par'ametros. Son importantes \Cmd{increase=no}, para que no le pregunte si desea aumentar el n'umero de 'arboles, y \Cmd{maxtrees=N} para dar como opci'on que guarde un n'umero de 'arboles (N).\\
\Cmd{exec}: para abrir la matriz de datos o archivos de instrucciones.\\
\Cmd{hsearch}: es la funci'on de b'usqueda de cladogramas. Las opciones de este comando son \Cmd{addseq=random}, lo que asegura que la entrada de datos para el 'arbol de Wagner es al azar.\\
\Cmd{swap=tbr/spr}: es el tipo de permutaci'on.\\
\Cmd{nreps=X}: indica el n'umero de r'eplicas que se har'a en la b'usqueda.\\ 
\Cmd{nchuck=X ckuckscore=1}: le permite usar la estrategia NONA, indicando c'uantos 'arboles desea por r'eplica. Si usted desea hacer solo permutaci'on de ramas del 'arbol previamente construido, la serie de instrucciones ser'ia \Cmd{search start=current chuckscore=no}.\\
Para guardar los 'arboles utilice el comando \Cmd{savetre}.\\
Con el comando \Cmd{contree} se generan los 'arboles de consenso. Estos deben ser guardados cuando se ejecuta el comado (par'ametro \Cmd{treefile=nombre}, donde nombre es el achivo.). Los m'etodos de medici'on de soporte implementados son el \textit{bootstrap} con el comando \Cmd{bootsptrap} y el \textit{jackknife} con \Cmd{jackknife}.\\
En caso de dudas, el programa posee una ayuda en linea, cons'ultela usando \Cmd{help} o escribiendo \Cmd{help comando} para un comando espec'ifico; para ver las opciones del comando use \Cmd{comando ?}.
\subsection{MrBayes}
\noindent
Autor: Ronquist et al., 2005.\\
Plataforma: Windows, MacOS, Unix.\\
Disponibilidad: gratuito.\\
http://mrbayes.net
\\
Es el programa m'as com'unmente usado para an'alisis bayesiano. Es gratuito y su interfaz es muy similar a la de PAUP*. Los principales comandos del programa son descritos en el capitulo de an'alisis bayesiano. El tutorial incluido es muy completo y viene incluido dentro del programa.\\
Una de las caracter\'isticas m\'as interesantes es la implementaci\'on de los ''modelos'' usados para caracteres morfol\'ogicos (Lewis, 2001), as\'i como el modelo de 'parsimonia' tal y como fue descrito por Tuffley \& Steel (1997) que se activa usando 

\Cmd {lset parsmodel=yes;}\\





%\section{Miscel\'aneos}
%\subsection{Component 2.0}
\noindent
Autor: Page, 1994. \\
Plataforma: Windows 16 bits (puede tener incompatibilidad con Windows XP, ME y superiores; instale soporte para 16 bits si desea correrlo); hay una versi\'on para Mac, pero con limitaciones.  (Funciona \textit{relativamente} bien en \Pname{wine} dentro de Linux)\\
Disponibilidad: el programa es gratuito y es distribuido desde la p'agina del autor; incluye el c\'odigo fuente en Pascal.

\url{http://taxonomy.zoology.gla.ac.uk/rod/cpw.html}


Este programa es 'util para c\'alculos de consensos; su principal aplicaci'on est'a en biogeograf\'ia.

%\subsection{ModelTest}
\noindent
Autores: Posada \& Crandall, 2001. La 'ultima vers'ion es Posada 2005.
\\Plataforma: Cualquiera (C\'odigo fuente en C).\\
Disponibilidad: Gratuito.\\
http://darwin.uvigo.es/people/dposada.html
\\
El programa utiliza como base los resultados de \Pname{PAUP*}, basado en una estimaci'on de la verosimilitud de un 'arbol producido por distancia de JC. El programa requiere un archivo de instrucciones para \Pname{PAUP*} distribuido con el programa, el cual debe ser ejecutado primero. Una vez se ha producido la salida de \Pname{PAUP*} (model.scores), esta se usa como base para los c\'alculos de los par\'ametros para \Pname{ModelTest}. La salida puede ser le'ida en un procesador de texto, donde se presentan los resultados de cada uno de los modelos examinados, tanto del test jer'arquico (hLRT) como del crit\'erio de informaci\'on de Akaike (AIC). Se pueden modificar mediante l\'inea de comandos las intrucciones dadas a Modeltest para el c\'alculo del modelo; de hecho, puede usar el programa como calculadora para $\chi ^2$.\\
Para plataforma Windows existe un manejador que le permite realizar todo el proceso con ayuda del rat\'on, y para MacOSX el programa mismo tiene una peque\~na interfaz.

%\input{}


