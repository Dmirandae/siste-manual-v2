\subsection{TNT}
\noindent
Autor: \cite{tnt}, plataforma: Windows, MacOS, Unix.

Disponibilidad: Gratis, es posible descargarlo desde \url{http://www.lillo.org.ar/phylogeny/tnt/}.

Para parsimonia, es el programa m\'as r\'apido que se ha desarrollado; incluye varios tipos de b\'usquedas especializadas, como la deriva y la fusi\'on de \'arboles.
Al igual que \Pname{NONA}, \cmd{proc} le permite abrir la matriz de datos o archivos de instrucciones. Para las b\'usquedas puede configurar los diferentes m\'etodos usando \cmd{ratchet}, \cmd{drift} y \cmd{mult}; al usar \cmd{?} puede obtener los par\'ametros, y con \cmd{=} puede modificarlos.
\cmd{Mult} puede usarse como la $"$central de b\'usqueda$"$, definiendo un n\'umero de r\'eplicas para una b\'usqueda de Wagner y las posteriores mejoras con ratchet y drift (seg\'un como est\'en configurados). Para correr simplemente escriba \cmd{mult: replic X;} para ejecutar el n\'umero de r\'eplicas que desee (X), con \cmd{tsave* nombre} se abre el archivo $"$nombre$"$ para guardar \'arboles. Gu\'ardelos con \cmd{save} y al final no olvide cerrar el archivo: \cmd{tsave/;}.

El programa cuenta con ayuda en l\'inea que puede ser consultada usando \cmd{help} o escribiendo \cmd{help comando} para un comando espec\'ifico. Este es \'unico programa que tiene directamente implementado el remuestreo sim\'etrico, adem\'as tiene implementado el soporte de Bremer, el soporte relativo de Bremer, soporte de Bremer dentro de los l\'imites del Bremer absoluto y puede calcular la cantidad de grupos soportados-contradichos (FC), con \cmd{resample} usted puede configurar la permutaci\'on. Con \cmd{subop} (tambi\'en en \Pname{NONA}) usted puede indicar el costo de los sub\'optimos, en diferencia de pasos con respecto al \'arbol \'optimo. \cmd{Bsupport} hace el c\'alculo del \'indice de Bremer, con un \cmd{*} calcula el valor relativo, o puede usar \cmd{Bsupport ]} para calcular el soporte relativo dentro de los l\'imites del absoluto. Los resultados se pueden almacenar usando algunas herramientas del lenguaje de macros\footnote{\url{www.lillo.org.ar/phylogeny/tnt/scripts/General_Documentation.pdf}}.


\subparagraph*{Versi\'on de men\'u}
En  Windows es bastante similar a \Pname{winClada}, pero esta mucho mejor organizada. Algunas funciones son:

\begin{itemize}
	\item \Gui{Settings}: Adem\'as de las diferentes opciones de macros, manejo de memoria, tambi\'en contiene los par\'ametros usados en el colapsado de ramas, consensos, y pesado impl\'icito.
	\item \Gui{Analyze}: Contiene los diferentes tipos de b\'usquedas y sus par\'ametros, el manejo de \'arboles sub\'optimos, y los remuestreos.
	\item \Gui{Optimize}: Permite ver y dibujar sinapomorf\'ias, mapear caracteres, y revisar estad\'isticos de caracteres y \'arboles (por ejemplo long o peso).
	\item\Gui{Trees}: All\'i se encuentran las entradas para el dibujo de \'arboles, el c\'alculo soporte de Bremer, realizar consensos y super-\'arboles, as\'i como \'arboles al azar, y el manejador de etiquetas de los nodos (\textit{tags}).
	\item \Gui{Data}: Aparte de la configuraci\'on de caracteres y terminales, posee un editor b\'asico de datos, que aunque muy simple, es extremadamente f\'acil de manejar, y m\'as directo que los editores basados en mostrar la matriz (Como \Pname{winClada}, o \Pname{Mesquite}).
\end{itemize}
