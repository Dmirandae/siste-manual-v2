%\chapter{Algunos comandos b\'asicos en R}

%\small
%\section
{Lectura b\'asica de una matriz (ape)}
%\noindent
$\#\#$
\\$\#\#$ cargamos la libreria ape
\\$\#\#$\\
\\
\Cmd{library(ape)}\\
\\
\\$\#\#$
\\$\#\#$ leemos datos alineados en formato tipo newick
\\$\#\#$ el alineamiento es solo por el ejercicio de mostrar
\\$\#\#$ lo que debe hacerse y los comandos respectivos
\\$\#\#$\\
\\
\Cmd{DNA <- read.dna(''alineado.phy'')}\\
\\
\\$\#\#$
\\$\#\#$ listado de tama\~no de las secuencias
\\$\#\#$ \\
\\
\Cmd{table(unlist(lapply(DNA, length)))}

%\section
{Test de modelos de evoluci\'on (ape)}
\label{sec:phytest}
\noindent
$\#\#$
\\$\#\#$ R y modelos via phyml 
\\$\#\#$
\\$\#\#$ cargamos la libreria ape
\\$\#\#$\\
\\
\Cmd{library(ape)}\\
\\
\\$\#\#$
\\$\#\#$  con phymltest probamos 28 modelos en phyml
\\$\#\#$  ape + R hacen todo el proceso
\\$\#\#$
\\$\#\#$  en linux cambie a
\\$\#\#$  execname = ''./phyml'', si es local o
\\$\#\#$
\\$\#\#$  execname = ''phyml'', si es global
\\$\#\#$
\\$\#\#$ en windows
\\$\#\#$ \\
\\
\Cmd{modelo<-phymltest(''alineado.phy'', format = ''sequential'', itree = NULL,exclude = NULL, execname = ''phyml.exe'', append = FALSE)}\\
\\
\\$\#\#$  
\\$\#\#$  use format = ''interleaved'' si aplica a su matriz 
\\$\#\#$
\\$\#\#$ resumen y plot 
\\$\#\#$ 
\\$\#\#$ con print los valores de AIC
\\$\#\#$ \\
\\
\Cmd{print(modelo)}
\\
\\$\#\#$ 
\\$\#\#$  con summary los valores del test jerarquico
\\$\#\#$ \\
\\
\Cmd{summary(modelo)}
\\
\\$\#\#$ 
\\$\#\#$ plot los valores de AIC
\\$\#\#$ \\
\\
\Cmd{plot(modelo, main = 'test de modelos para ML, usando PhyML')}
\\
