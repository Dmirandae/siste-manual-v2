\chapter{Alineamiento de ADN}
\section*{Introducci\'on}
\label{ch:alinear}

Aunque los datos moleculares pueden proporcionar grandes cantidades de caracteres, presentan retos 
en la asignaci\'on de homolog\'ias al tratarse exactamente de las mismas bases en todas las secuencias; adem\'as 
en muchas ocasiones las secuencias tienen un tama\~no desigual. \index{alineamiento!definici\'on}
Para resolver estos problemas se han ideado los m\'etodos de alineamiento, que buscan recuperar las posibles homolog\'ias 
dentro de diferentes secuencias, al utilizar una matriz de transformaciones, entre las cuatro bases y los gaps o 
InDels (inserci\'on/eliminaci\'on); as\'i, colocando gaps las dos secuencias son equivalentes en costos y  posiciones
 para las bases; por ejemplo, compare la secuencia 1x con cualquiera de los alineamientos 2.
\footnote{Ejemplo modificado de Siddall, http://research.amnh.org/~siddall/methods.}\\
\\
\begin{small}
1x: \cmd{t T C C g A A T T g g c t a c T T C C g A A t T T g G}\\
2x: \cmd{t C G A T T G C C A C T C G A T T G}\\
2a: \cmd{t - - C g A - - T g c c t a c T - C - g A - t T - g G}\\
2b: \cmd{t - C - g A - T - g c c t a c - T - C g A - t - T g G}\\
2c: \cmd{t - C - g - A T - g c c t a c - T - C g - A t - T g G}\\
\end{small}
\\
Ya que se necesitan m\'as de tres secuencias para un an\'alisis filogen\'etico, es necesario el alineamiento simult\'aneo de 
m\'ultiples secuencias. El primer m\'etodo usado es el alineamiento m\'ultiple que usa un \'arbol \textbf{gu\'ia} con las 
secuencias en los terminales; a partir de este \'arbol se hace el primer alineamiento para el par m\'as 
cercano\footnote{En realidad es el par m\'as similar, ya que se usa una t\'ecnica de distancia para obtener el \'arbol gu\'ia.}, 
luego optimiza los gaps, posteriormente se alinea con la siguiente terminal que sea la m\'as cercana y el proceso se repite 
hasta llegar a la base; la idea es parecida a la optimizaci\'on en un an\'alisis filogen\'etico.\\
Una nueva idea, usada principalmente por cladistas, es la optimizaci\'on directa \citep{Wheeler1996}, donde el 
objetivo no es construir el alineamiento, sino directamente el cladograma; en este caso, se hacen los alineamientos locales 
(como en el alineamiento m\'ultiple) y los gaps son considerados transformaciones, es decir se colocan temporalmente en 
los nodos. \cite{Wheeler1996} argumenta que implementa una idea de homolog\'ia din\'amica acorde con el an\'alisis filogen\'etico. 
En la optimizaci\'on de estados fijos \citep{Wheeler1999} se usa la secuencia completa, y la matriz de transformaciones s\'olo 
sirve para asignar costos, pero no estados en el nodo. \index{alineamiento!m\'ultiple} Aunque es intuitivamente muy 
interesante, no se ha usado en la pr\'actica. Cualquiera que sea la idea para realizar los an\'alisis moleculares, los 
resultados dependen de la matriz de transformaci\'on inicial; \cite{Wheeler1995} desarroll\'o una metodolog\'ia para comparar 
los par\'ametros de las transformaciones, conocida como $"$an\'alisis de sensibilidad$"$.
\index{alineamiento!an\'alisis de sensibilidad}
Bajo esta idea, se hacen diferentes an\'alisis con diferentes matrices de transformaci\'on, y se selecciona aquella que 
maximice alg\'un criterio previamente seleccionado (ya sean las topolog\'ias, como el costo de las transformaciones); hasta 
ahora, esta idea solo se ha utilizado en el contexto de an\'alisis con m\'ultiples genes o entre genes y morfolog\'ia 
simult\'aneamente.

\section*{T\'ecnicas}

Si desea hacer un alineamiento m\'ultiple r\'apido use preferencialmente \Pname{MUSCLE} sobre \Pname{CLUSTAL}, pero si el objetivo es la filogenia sugerida por las secuencias es mejor usar  \Pname{POY}, ya que  tiene en cuenta el \'arbol final. Puede hacer una  exploraci\'on en \Pname{CLUSTAL} y posteriormente llevar sus marcos como archivos separados para ser analizados con \Pname{POY}.

Algunos programas (como \Pname{CLUSTAL}) usan un solo \'arbol gu\'ia derivado del c\'alculo de la distancia entre las 
secuencias; como todos los m\'etodos basados en distancias, es dependiente del orden de entrada de los datos; \Pname{MUSCLE} aunque parte de un \'arbol gu\'ia de  distancia, revisa el resultado del alineamiento, primero a nivel goblal y luego a nivel local;  \Pname{POY} construye distintos \'arboles y prefiere el alineamiento que sugiere el \'arbol de menor costo. Aunque sea lento, \Pname{POY} produce mejores resultados (en cuanto a la calidad del alineamiento con miras a evaluar la filogenia) que \Pname{CLUSTAL}. \cite{Giribet1999} recomiendan que si se usa \Pname{CLUSTAL}, se prueben diferentes secuencias de entrada de los datos, es importante recalcar que el \'arbol usado por \Pname{CLUSTAL} es s\'olo un \'arbol para optimizar las secuencias, no un \'arbol filogen\'etico como tal; 
mientras que el \'arbol generado por \Pname{POY} es un \'arbol de alineamiento y a la vez es un \'arbol filogen\'etico. Dado que el objetivo de \Pname{POY} es el an\'alisis simult\'aneo, a lo largo de este libro se usar\'a \Pname{POY} como otro programa m\'as de an\'alisis. Aun as\'i, \Pname{POY} puede producir los alineamientos impl\'icitos sugeridos por cada \'arbol (as\'i se lo utilizar\'a en esta pr\'actica), por lo que es posible usar este resultado como entrada para programas de b\'usquedas que usen secuencias alineadas, de usar los alineamientos, los costos usados para el alineamiento deber\'ian ser los mismos usados para el an\'alisis filogen\'etico. 

Al asignar los costos, hay que tener en cuenta la desigualdad triangular,  es decir, que el costo de una transformaci\'on nunca puede ser mayor al de una transformaci\'on equivalente, pero que tome otros estados,  si las transiciones tienen un costo de 3 y las transversiones de 1, la transformaci\'on A$\rightarrow$ G, tendr\'ia un costo de 3, pero si se hace de la forma A$\rightarrow$ T$\rightarrow$ G, el costo ser\'ia de 2 (dos transversiones). Esto har\'ia que a los nodos se les pudiesen asignar estados no observados en los terminales.


Para facilitar el alineamiento como las asignaciones de homolog\'ia, al usar \Pname{POY} se pueden (y preferencialmente se deben) dividir las secuencias analizadas en peque\~nas secuencias o marcos; definidos mediante iniciadores universales, estructura secundaria o motivos conservados. Cuando las secuencias son muy desiguales en estos segmentos, muchas veces se prefiere eliminarlas a analizarlas (o se hace un an\'alisis exploratorio que incluya esas secuencias).


\begin{enumerate}
	\item En \Pname{CLUSTAL/MUSCLE}:
		\begin{enumerate}
			\item  Abra con un editor los archivos \Fname{datos.ou.dat}.
			\item   Ejecute \Pname{CLUSTAL} con los par\'ametros predefinidos cual es la relaci\'on transici\'on/tranversi\'on definida?
			\item   Modifique los par\'ametros de tal forma que obtenga una relaci\'on de ts/tv de (a) 1/2 (b) 1/5.
			\item   Ejecute nuevamente \Pname{CLUSTAL} con estas modificaciones en los par\'ametros.
			\item   Cambie los par\'ametros dando a los gaps un costo del doble del costo de las transversiones.
			\item   Ejecute \Pname{MUSCLE} con los par\'ametros predefinidos. ?`cu\'al es la relaci\'on transici\'on/tranversi\'on definida?
			\item   Repita el paso 5 y ejecute nuevamente \Pname{MUSCLE} con los nuevos par\'ametros.
			\item   Guarde el alineamiento en un archivo de texto en formato \Pname{FASTA}, \Pname{PHYLIP} o \Pname{NEXUS}.
		\end{enumerate}

	\item \Pname{POY} cuenta con una interfaz de usuario que es relativamente r\'apida de dominar. Para los diferentes tipos de costos, el comando m\'as importante es \cmd{transform()}. Por ejemplo, con \cmd{transform(tcm(1,2))} se da peso de 1 a las sustituciones y de 2 a los gaps o a los datos morfol\'ogicos (est\'aticos). Matrices de transformaci\'on m\'as complejas pueden elaborarse y luego leerse con ese mismo comando. Una de las grandes ventajas de \Pname{POY} es que puede ser usado en un {\textit{cluster}} de computadoras.
	
	\item En \Pname{POY}:
		\begin{enumerate}
	  	 \item  Abra los datos en \Pname{POY} usando 
	  	 \Cmd{read($"$datos.fas.dat$"$)}
	  	 coloque todos los costos iguales y realice la b\'usqueda con: 
	  	 \Cmd{transform (tcm(1,1)); build( 100); 	  	 
	  	 select( unique);swap();select()}
		 \item Transforme sus secuencias de homolog\'ia din\'amica a est\'atica, use \cmd{transform(static\_aprox)}.
		 \item Guarde el alineamiento en un archivo de texto usando 
		 \Cmd{report($"$alin.txt$"$,phastwinclad)}
		 \item Revise el alineamiento con \Pname{Winclada}.
		 \item Repita los pasos anteriores incluyendo la matriz de datos morfol\'ogicos.
		 \item Repita los pasos anteriores con un alineamiento tipo FSO use \cmd{transform(fixed\_states)}.
		 \item Pese la matriz morfol\'ogica 2 veces el valor de los datos moleculares usando \cmd{transform((static,weight:2))}.
		 \item Cambie los par\'ametros de tal manera que los gaps valgan el doble que las sustituciones.
		 \item Elabore una matriz de costos para obtener relaciones de ts/tv de  1/2 o  1/5, l\'eala usando \cmd{transform((all, tcm:($"$costos.txt$"$))}, y genere el alineamiento.
		 \item Haga un an\'alisis de sensibilidad usando las matrices \Fname{datos.fas.dat} y \Fname{datos2.fas.dat} usando los mismos costos para los dos conjuntos de datos y con costos diferenciales para cada tipo de datos, revise el ap\'endice \ref{ch:usoPOY}, p\'agina \pageref{ch:usoPOY}.
		 \item Haga un an\'alisis tradicional (build, select, swap, select) y comp\'arelo con un an\'alisis sin seleccionar los \'arboles para swap(=build, swap, select).
		 \pregunta{?`Qu\'e efecto tiene el \cmd{select} intermedio?}
		 \item La secuencia analizada div\'idala en al menos tres marcos y repita el proceso, eval\'ue los tiempos usados en cada uno de los procesos.
		  \pregunta{?`Puede hacer el an\'alisis del punto anterior con alg\'un comando de POY?, b\'usquelo y una vez lo haga repita el proceso pero fraccionando la secuencia usadas.}
		  \item Repita los tres pasos anteriores usando ML.

           \pregunta{ 
             \begin{itemize}
               \item  ?`C'omo hace \Pname{POY} la reconstrucci\'on bajo  ML? revise  el manual 
               % como \url{https://www.dropbox.com/s/jz132uakl95d3zh/Pedraza\_et\_al\_2013.pdf}
               \item ?`Las topolg'ias obtenidas v\'ia parsimonia son iguales (o no) a las generadas por ML?
               \item ?`Los alineamientos impl\'icitos son equivalentes? 
               use los costos de los alineamientos y los caracteres informat\'ivos evaluados en \Pname{winclada}.
              \end{itemize}
            }

          \item  Repita todo el proceso usando FSO.
          
          \pregunta{?`Es posible hacer FSO usando ML?}

        \end{enumerate}

\end{enumerate}

\preguntaGral{
\begin{itemize}
	\item Compare los alineamientos de \Pname{MUSCLE} y \Pname{CLUSTAL}, al modificar los distintos par\'ametros, ?`Son similares los resultados?
	\item Compare los \'arboles de \Pname{MALIGN}, \Pname{POY} y los obtenidos en un an\'alisis clad\'istico con el programa de su  preferencia. ?`Son similares los resultados?
	\item Compare sus resultados con los de sus compa\~neros. ?`C\'uales son los costos de los \'arboles (reportados por \Pname{POY}) y las topolog\'ias?
	\item Dados los diferentes costos usados en el an\'alisis simult\'aneo de morfolog\'ia y datos moleculares,  ?`cu\'al cree usted que es el resultado \'optimo y por qu\'e?
	\item Existe una disputa sobre una relaci\'on entre los juegos de costos y el soporte. Dados sus conocimientos, escriba un peque\~no ensayo donde indique sus ideas, su posici\'on y sus argumentos en esta discusi\'on.
\end{itemize}
}


\section*{Literatura recomendada}
\noindent
\cite{Edgar2004} [Compara los alineamientos derivados de MUSCLE y CLUSTAL].

\cite{Frost2001} [Un art\'iculo emp\'irico sobre el an\'alisis de sensibilidad].

\cite{Giribet2003} [Revisa la exploraci\'on de los resultados del an\'alisis de sensibilidad vs. soporte de clados].

\cite{Giribet1999} [Uno de los pocos art\'iculos que discute expl\'icitamente el tema de los gaps].

\cite{Wheeler1995} [La propuesta del an\'alisis de sensibilidad para la asignaci\'on de par\'ametros en los alineamientos].

\cite{Wheeleretal2006} [Una gu\'ia completa para POY].
%Pedraza et al 2013
