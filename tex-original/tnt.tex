\subsection{TNT}
\noindent
Autor: Goloboff et al., 2009.\\ % colocar cita al paper de cladistics
Plataforma: Windows, MacOSX, Linux.\\
Disponibilidad: gratuito.\\
http://www.lillo.org.ar/phylogeny/tnt/.


Para parsimonia, es el programa m\'as r\'apido que se ha desarrollado; incluye varios tipos de b\'usquedas especializadas, como la deriva y la fusi'on de \'arboles, una de sus fortalezas es el lenguaje de macros que permite automatizar an\'alisis repetitivos y que contengan m\'ultiples reglas de decisi\'on.


Al igual que en \Pname{NONA}, \cmd{proc} le permite abrir la matriz de datos o archivos de instrucciones. Para las b\'usquedas puede configurar los diferentes m\'etodos usando \cmd{ratchet}, \cmd{drift} y \cmd{mult}; al usar \cmd{?} puede ver c\'uales son los parametros, y con \cmd{=} puede modificarlos.

\cmd{Mult} puede usarse como la ''central de b'usqueda'', definiendo el n\'umero de r\'eplicas para una b\'usqueda de Wagner y las posteriores mejoras con ratchet y drift (seg'un como est'en configurados). Para correr simplemente escriba \cmd{mult: replic N;} para ejecutar el n'umero de r\'eplicas que desee (N).\\
Con \cmd{tsave* nombreArchivo} se abre el archivo ''nombreArchivo'' para guardar \'arboles, los cuales se almacenan con \cmd{save}, al final no olvide cerrar el archivo: \cmd{tsave/;}.
El programa cuenta con ayuda en l'inea que puede ser consultada usando \cmd{help} o escribiendo \cmd{help comando} para un comando espec'ifico.

Este es \'unico programa que tiene directamente implementado el remuestreo sim\'etrico, el soporte (absoluto) de Bremer, el soporte relativo de Bremer, el soporte de Bremer dentro de los l\'imites del Bremer absoluto y puede calcular la cantidad de grupos soportados-contradichos (FC). Se pueden configurar los par\'ametros de la permutaci\'on con \cmd{resample}, con \cmd{subop} (tambi'en en \Pname{NONA}) usted puede indicar la longitud de los sub'optimos, en diferencia de pasos con respecto al \'arbol \'optimo. \cmd{Bsupport} hace el c\'alculo del \'indice de Bremer, con un \cmd{*} calcula el valor relativo, o puede usar \cmd{Bsupport ]} para calcular el soporte relativo dentro de los l\'imites del valor absoluto. Los resultados se pueden guardar directamente sobre la topolog\'ia.\\
