\chapter{Soporte en parsimonia}
\section*{Introducci'on}
\index{Soporte!parsimonia}
\label{ch:soporte.pars}
Luego de obtener los cladogramas, para algunos es importante saber qu'e tan fuerte es la evidencia que da soporte a un nodo. Es importante diferenciar la \textbf{evidencia que soporta} el nodo, es decir las transformaciones (sinapomorf'ias), de la \textbf{fuerza del soporte} es decir la diferencia entre las agrupaciones encontradas con posibles agrupaciones alternativas. Aqu'i se usar'a soporte en ese segundo sentido.\\
Existen diferentes m'etodos para medir el soporte, unos basados en remuestreo al azar de los caracteres y otros en el uso de 'arboles sub'optimos. En los m'etodos de remuestreo se toma la matriz original, se perturba, y luego se analiza. Al final, se hace un consenso de la mayor'ia, donde la frecuencia de aparici'on de cada nodo (reportada como porcentaje) da una idea de la cantidad de evidencia favorable para este. Los m'etodos de remuestreo m'as populares (especialmente en los an'alisis moleculares) son el \textbf{Bootstrap}, donde se construye una matriz del mismo tama'no de la original usando caracteres de la matriz original tomados al azar. En el \textbf{Jackknife} y en la \textbf{remuestreo sim'etrico} cada car'acter puede ser borrado independientemente (jac y sim) o repesado (sim). En estos dos m'etodos, la probabilidad de seleccionar cada car'acter es independiente de los dem'as y cada car'acter s'olo es muestreado una vez. Aunque eso permite que los caracteres no informativos y los autapom'orficos no afecten el resultado, estos se hacen dependientes de la probabilidad usada para muestrear cada car'acter.\\
Otra forma de medir el soporte es utilizar 'arboles sub'optimos (Bremer, 1994). En este m'etodo se hace un consenso estricto con esos 'arboles sub'optimos, y eventualmente, cuanto m'as sub'optimos sean los 'arboles, m'as nodos de los encontrados en los 'arboles 'optimos desaparecer'an. La idea, entonces, es contar la diferencia de longitud necesaria para que el nodo colapse.\\
Las medici'ones del soporte con remuestreo (usando consenso de la mayor'ia) o con cladogramas sub'optimos (usando soporte de Bremer) s'olo tienen en cuenta la cantidad de evidencia favorable. Un nodo puede estar bien soportado (por un car'acter no contradicho), y es posible que nodos que aparecen con mejor soporte, tengan muchos caracteres a favor, pero tambi'en muchos en contra. Para resolver este problema, en remuestreo se ha elaborado el 'indice GC (Goloboff et al., 2003), que es la diferencia de la cantidad de veces que aparece el grupo (el soporte usual) y el n'umero de veces que aparece el mejor grupo laternativo. Para el soporte de Bremer relativo (Goloboff \& Farris, 2001) la idea es similar: comparar la longitud de los 'arboles que favorecen el grupo y los que lo contradicen.\\
El concepto de soporte puede extenderse al an'alisis de particiones. En este caso, se mide qu'e tan soportado est'a un nodo en una parte de los datos (una partici'on, que puede ser, por ejemplo, uno de los diferentes genes analizados). Si el nodo esta presente en el consenso de los 'arboles m'as parsimoniosos de la partici'on, el soporte se halla de la forma convencional; si por el contrario el nodo no est'a presente, se busca el 'arbol m'as parsimonioso que contenga el nodo, y la diferencia de pasos entre este 'arbol y el 'arbol m'as parsimonioso de la partici'on es el soporte de Bremer negativo del nodo (es negativo, indica qu'e tan contradicho es el nodo). Los soportes por particiones permiten detectar cu'ales son las particiones que sugieren el nodo y cuales lo contradicen. De esta forma, no solo se mide el soporte, sino que es posible observar la congruencia por nodos entre las particiones. Aunque al sumar los valores de Bremer de las particiones el resultado puede ser igual al valor del soporte de Bremer, esto no siempre sucede.
\section{T'ecnicas}
La medici'on de soporte utiliza conjuntos de 'arboles, as'i que el problema en este caso es c'omo conseguir ese conjunto de 'arboles. Las perturbaciones de la matriz est'an implementadas en casi todos los programas, por lo que el proceso puede realizarse de forma autom'atica. En algunos casos los parametros de la perturbaci'on son dif'iciles de aclarar. En \textit{bootstrap} no existe problema, puesto que se toma de la matriz un car'acter al azar hasta completar una matriz del mismo tama~no de la original, as'i no existe diferencia entre las diferentes fuerzas de permutaci'on. El problema es que en las matrices hay caracteres no informativos que pueden sesgar la matriz producto de la permutaci'on. Otro inconveniente es que la mayor'ia de los datos moleculares son secuencias sucesivas de caracteres, y los morfol'ogicos no es f'acil entender c'omo est'an relacionados. El m'etodo exige una distribuci'on homog'enea de la informaci'on y que el muestreo sea al azar.\\
Con \textit{jackknife} y permutaci'on sim'etrica, cada car'acter es alterado independiente de los dem'as, con lo que autapomorf'ias y caracteres no informativos no influyen en la distribuci'on final. Adem'as, los m'etodos basados en permutaciones exigen que los resultados de la matriz permutada sean confiables. Es decir, b'usquedas estrictas. Esto hace que los m'etodos consuman mucho tiempo. Este problema se ha solucionado usando muchas b'usquedas muy superficiales, y usando el consenso estricto (o uno muy fuerte) de las b'usquedas independientes (ver Farris et al., 1996; Goloboff \& Farris, 2001), aunque eso podr'ia disminuir la frecuencia de los nodos con soportes bajos (en el dado caso de que esos sean de inter'es).\\
Es importante notar que cuando se hacen remuestreos con \Pname{POY}, en el caso de los datos moleculares, los fragmentos completos de ADN son usados como caracteres, y no cada base particular (como se hace tradicionalmente con una matriz previamente alineada), por lo que jackknife puede eliminar fragmentos completos de datos moleculares.\\
El soporte de Bremer utiliza 'arboles sub'optimos. El problema es que muchos programas no lo tienen definido expl\'icitamente, por lo que es necesario recurrir a t'ecnicas externas para poder hacer la medici'on. Adem'as, la mayor'ia de 'arboles sub'optimos se buscan usando como fuente uno de los 'arboles 'optimos y luego permutando las ramas y reteniendo los cladogramas que cumplan con la longitud m'axima especificada. Esto genera dos problemas: el primero, es que la mayor parte de los 'arboles pueden pertenecer a un mismo \textbf{vecindario} o \textbf{isla} de 'arboles; el segundo es que muchos de los cladogramas que pueden colapsar el nodo, en realidad son cladogramas que no presentan evidencia de agrupamiento, es decir, no hay sinapomorf'ias en los nodos. Otra forma de realizar el soporte de Bremer es hacer b'usquedas del 'arbol m'as parsimonioso que no contenga el grupo al cual se desea medir el soporte. Una ventaja de esto, es que proporciona una medici'on directa, aun para nodos no presentes en el 'arbol m'as parsimonioso. La desventaja m'as notoria es que hay que realizar muchas b'usquedas para medir el soporte de cada nodo.
\section{Materiales}
\noindent
Matriz de datos (datos.soporte.pars.dat).
Matriz de datos (datos.particiones.dat).
Macro para \Pname{TNT} (brempart.run)
\section{M'etodos}
\noindent
\textbf{En \Pname{WinClada}, \Pname{PAUP*}, \Pname{POY} y \Pname{TNT}:}\\
(1) Abra la matriz de datos, realice \textbf{bootstrap} y \textbf{jackknife} (con los valores por omisi'on y usando corte del 36\%). Haga b'usquedas simples para obtener los resultados durante la pr'actica (50 r'eplicas). Compare sus resultados con los de sus compa~neros.\\
\textbf{En \Pname{TNT} y \Pname{NONA}:}\\
(2) Abra la matriz de datos, b'usque el 'arbol m'as parsimonioso.\\
(3) Retenga 1000 'arboles y acepte sub'optimos hasta  5 pasos m'as largos. 
% error en nuemro de arboles de wagner corregido 230409
Haga 500 r'eplicas de Wagner sin permutar ramas, reteniendo solo un 'arbol por r'eplica. Luego llene la pila de 'arboles haciendo permutaci'on de ramas, use \Cmd{bbreak} en \Pname{TNT}, y en \Pname{NONA} \Cmd{max*;}.\\
(4) Con los 'arboles obtenidos calcule el soporte de Bremer, absoluto y relativo.\\
(5) Repita (3) y (4), pero reteniendo 'arboles 7 pasos m'as largos.\\
\textbf{En \Pname{POY}:}\\
(6) Calcule el soporte de Bremer.\\
\textbf{En \Pname{TNT}:}\\
(7) Abra la matriz datos.particiones.dat en un editor de texto. Note al final la presencia de la instrucci'on \Cmd{blocks}, esta instrucci'on le permite definir grupos de caracteres en \Pname{TNT}.\\
(8) Abra la matriz con \Pname{TNT}, mire cuantos bloques de datos est'an definidos y ejecute el macro brempart.run, con el n'umero de particiones definido (por ejemplo para 3 particiones, \Cmd{run brempart.run 3;}). El macro le reporta para cada nodo el soporte de Bremer para los datos completos, y para cada partici'on definida. Para ver los valores abra la matriz y el \'arbol de salida (BremPart.txt) y escriba \Cmd{ttags;}. Compare los valores.
\subsection{Programas}
\noindent
\Pname{WinClada}, \Pname{PAUP*}, \Pname{POY} y \Pname{TNT}.
\subsection{Comandos}
En \Pname{NONA} y \Pname{TNT} el soporte de Bremer se obtiene con \Cmd{bsupport}, \textbf{debe} haber calculado previamente los 'arboles  sub'optimos con \Cmd{subop} y la respectiva b'usqueda. Si usa la instrucci'on con un asterisco \Cmd{bsupport*}, el soporte calculado es el relativo en forma de porcentaje; tambi'en puede usar \Cmd{bsupport [;} o \Cmd{bsupport ];}. \Pname{TNT} usa el comando \Cmd{resample} para los diferentes m'etodos de soporte basados en permutaci'on de la matriz. Si obtiene soportes relativos superiores a 100\%, eso indica que los 'arboles sub'optimos no son suficientes para evaluar el nodo, por lo que debe repetir el c'alculo pero con un \Cmd{subop} m'as grande; repita el proceso al menos tres veces para evitar que los 'arboles sub'optimos sean muestreos de unas pocas ''islas'' de 'arboles.\\
\Pname{POY} implementa el soporte de Bremer con 
\Cmd{calculate\_support()} usando los 'arboles m'as cortos que no contengan cada clado. Con diferentes par'ametros se implementan el bootstrap y el jackknife, en esos casos es bueno incluir el tipo de b'usquedaen general simpledurante el remuestreo. Por ejemplo 

\Cmd{calculate\_support(bootstrap:100, build(10),swap())} 

calcula 100 replicas de bootstrap, y en cada una se construyen 10 arboles de Wagner, mejorados con permutado de ramas, y 

\Cmd{calculate\_support(jackknife (resample: 100, remove:33.3), build(10),swap())}

 calcula 100 replicas de jackknife eliminando el 33.3\% de los caracteres (por omisi'on se elimina 36\%) y las b'usquedas son iguales a las del ejemplo de bootstrap.\\
En \Pname{PAUP*} solo se hacen permutaciones sobre la matriz; con \Cmd{bootstrap} se configura y ejecuta el \textit{bootstrap}, mientras que con  \Cmd{jackknife} se hace lo propio para \textit{jackknife}; recuerde que \Pname{PAUP*} es m'as lento que \Pname{NONA}/\Pname{TNT}, por lo que los comandos de b'usqueda y el n'umero de r'eplicas deben ser acordes con sus tiempos.
\section{Preguntas}
\subsection{Pr'actica}
\noindent
Compare con sus compa'neros sus resultados de soporte basado en permutaciones y en soporte de Bremer. ?`Usted sugeriri'a o no el uso de soporte relativo? Explique las razones de su escogencia.\\
?`Los nodos m'as soportados son los mismos? ?`Existe correlaci'on entre los diferentes m'etodos?
\subsection{Generales}
\noindent
Escriba un peque~no ensayo con sus puntos de vista sobre los diferentes m'etodos de medici'on de soporte, destacando tanto los puntos positivos como los negativos de cada metodolog'ia.\\
Indique cu'ales m'etodos, y por qu'e, preferir'ia si trabaja: (1) solo con datos morfol'ogicos, (2) solo con datos de ADN, (3) una matriz combinada.\\
?`Cree usted que pueden existir m'etodos m'as eficaces para medir el soporte? De ser as'i, escriba un peque~no ensayo sobre las cualidades esperables teniendo en cuenta los distintos tipos de datos.\\
?`Qu'e es lo que usted desear'ia y esperar'ia de un m'etodo que mida el soporte?\\
\Pname{POY} Al remover fragmentos completos de ADN, ?`puede influenciar los c'alculos de soporte usando m'etodos de remuestreo?
\section{Literatura recomendada}
\noindent
Goloboff \& Farris, 2001 [Presentaci'on del soporte relativo de Bremer].\\
Goloboff et al., 2003 [Propuesta del muestreo sim\'etrico as\'i como de diversas medidas de soporte basadas en remuestreo].\\
Grant \& Kluge, 2003 [Una cr'itica extensa a los diferentes formas de medici'on de soporte].\\
Ram'irez, 2005 [Excelente revisi'on de los diferentes m'etodos para medir el soporte, incluyendo los m'etodos m'as recientes].
